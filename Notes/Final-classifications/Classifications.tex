\documentclass[10pt,fleqn]{article}
\usepackage{/home/clair/Documents/mystyle}

%----------------------------------------------------------------------
% reformat section headers to be smaller \& left-aligned
\titleformat{\section}
	{\normalfont\bfseries}
	{\thesection}{1em}{}
	
\titleformat{\subsection}
	{\normalfont\bfseries}
	{\thesubsection}{1em}{}
	
%----------------------------------------------------------------------
\renewcommand{\todo}[1]{
	\begin{minipage}{\textwidth}
	\textcolor{black}{
					\begin{tabular}{p{0.01\textwidth}p{0.95\textwidth}}
						$\square$ & #1
					\end{tabular}
					}
	\end{minipage}
}


%======================================================================

\begin{document}

Several proposed classification approaches...

\begin{description}
\item[Mean-valued:] pixels are classified only according to their value. Graduations of locally-bright/dim to hot/dead are included, so severity of defect can be traced.

\item[Mean-valued grey/black:] as above, but only the grey and black imges are used for classification - the white image is completely ignored, so we use only a single illuminated image.

\item[Qualitative:] pixels are classified according to their behaviour: bright in black image, or grey?

\item[Feature-based:] pixels may be subsumed by a feature rooted in an adjacent pixel (such as a cluster or line).
\end{description}

\vspace{15pt}

Also several means of assessing quality of each approach:

\begin{description}

\item[State stability:] do pixels tend to remain in the same state (suggesting a genuine difference in state) or move between adjacent states (suggesting an inappropriate threshold?) - both mean transition rate and standard deviation of transition rate

\item[Path stability:] if we group pixels according to the path taken through the proposed state space, are there a lot of paths? Do pixels tend to fluctuate between a number of states (suggesting poor thresholding) or to progress from less to more severe?

\item[Observed vs expected prevalence:] since all pixels were observed at each acquisition, we can compare the model's predictions with the observed data (expected vs observed counts in each category) - either numerically or with plots of O/E over time. Could use $\chi^2$ test of goodness of fit?
\end{description}

\vspace{15pt}

\todo{Compare classifications obtained using each set of criteria: does much change?}

\todo{check if transition rates are constant: if proportion changing at each step is plotted in real time, what is the trend? Do proportions remain constant over time, or is there evidence of an increase?}


%\newpage
%\printbibliography
\end{document}