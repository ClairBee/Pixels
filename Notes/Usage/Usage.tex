\documentclass[10pt,fleqn]{article}
\usepackage{/home/clair/Documents/mystyle}

%----------------------------------------------------------------------
% reformat section headers to be smaller \& left-aligned
\titleformat{\section}
	{\normalfont\bfseries}
	{\thesection}{1em}{}
	
\titleformat{\subsection}
	{\normalfont\bfseries}
	{\thesubsection}{1em}{}
	

    
%======================================================================

\begin{document}

\begin{figure}[!ht]
\caption{Plots of estimated usage (in arbitrary units, obtained from CT profile data); each point represents a set of images obtained for computed tomography, with the $y$-axis denoting the product of current, voltage, exposure time, number of projections, and frames per projection. Red dotted lines indicate acquisition dates for the Inside-Out project. The omission of single-image acquisitions is considered unlikely to make a substantial difference to the estimated usage figures, with a typical acquisition of 60 calibration images adding 0.17 arbitrary units on this scale.\\
\footnotesize{The gold line shows a fitted linear regression  model, the turquoise a linear model with quadratic time. The difference between the two is small, suggesting that average usage is fairly constant; however, there are clearly periods of heavier use, which may be related to higher rates of pixel damage.}}

\centering

\begin{subfigure}[t]{0.49\textwidth}
\caption{Detector usage}
\includegraphics[scale=0.45]{./fig/usage-vs-time}
\end{subfigure}
%
\begin{subfigure}[t]{0.49\textwidth}
\caption{Cumulative usage}
\includegraphics[scale=0.45]{./fig/cum-usage-vs-time}
\end{subfigure}
%

%
%\begin{subfigure}[t]{0.49\textwidth}
%\caption{Moving average usage over 10 preceding scans}
%\includegraphics[scale=0.45]{./fig/ma-usage-vs-time}
%\end{subfigure}

\end{figure}


\begin{figure}[!ht]
\caption{Percentage of detector covered by defective pixels: plotted against time and cumulative usage. Again, a linear model is shown in gold, quadratic model in turquoise. The rate of increase seems to more closely reflect the age of the detector than the recorded amount of usage.}
\centering

\begin{subfigure}[t]{0.49\textwidth}
\caption{Vs time}
\includegraphics[scale=.45]{./fig/defects-vs-time}
\end{subfigure}
%
\begin{subfigure}[t]{0.49\textwidth}
\caption{Vs cumulative usage}
\includegraphics[scale=.45]{./fig/defects-vs-usage}
\end{subfigure}

\end{figure}


%\begin{figure}[!ht]
%\caption{Percentage of detector covered by defective pixels as a function of cumulative usage. The relationship appears to be largely linear, apart from a jump in the number of pixels identified as locally bright in the last-but-one image (acquired on 16-03-14). This jump coincides with a change in software, which led to the temporary loss of the gain setting (30dB) that we had been using - which was reinstated by March 23rd, well before the next set of images was acquired.\\
%\footnotesize{A look at plots of mean vs variance shows that the grey images from that date have a higher mean SD than might be expected, given their mean values, but not by much.}}
%
%%\begin{subfigure}[t]{0.49\textwidth}
%%\caption{All defective pixels, including lines but not including edge pixels or screen spots}
%%\includegraphics[scale=0.45]{./fig/cum-usage-vs-bp-prop}
%%\end{subfigure}
%%
%%\vspace*{10pt}
%
%\begin{subfigure}[t]{0.49\textwidth}
%\caption{All defective pixels by type, including lines but not including edge pixels or screen spots}
%\includegraphics[scale=0.45]{./fig/cum-usage-vs-bp-prop-by-type}
%\end{subfigure}
%%
%\begin{subfigure}[t]{0.49\textwidth}
%\caption{All defective pixels by type, focusing on the smallest categories}
%\includegraphics[scale=0.45]{./fig/cum-usage-vs-bp-prop-by-type-zoom}
%\end{subfigure}
%
%\end{figure}

\begin{figure}[!ht]
\caption{Percentage of detector covered by defects over time. \\
There is no obvious link between high usage and increased detector damage, even when splitting out different subsets (eg. high power, long exposure etc)}

\begin{subfigure}[t]{0.49\textwidth}
\caption{Damaged pixels}
\includegraphics[scale=0.45]{./fig/usage-vs-defect-prop}
\end{subfigure}
%
\begin{subfigure}[t]{0.49\textwidth}
\caption{Spots on beryllium screen}
\includegraphics[scale=0.45]{./fig/usage-vs-screen-spot-prop}
\end{subfigure}

\end{figure}


\end{document}
