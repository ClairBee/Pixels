\documentclass[10pt,fleqn]{article}
\usepackage{/home/clair/Documents/mystyle}

 \usepackage{pdflscape}
 
 \newcolumntype{M}[1]{>{\centering\arraybackslash}m{#1}}
 
%----------------------------------------------------------------------
% reformat section headers to be smaller \& left-aligned
\titleformat{\section}
	{\normalfont\bfseries}
	{\thesection}{1em}{}
	
\titleformat{\subsection}
	{\normalfont\bfseries}
	{\thesubsection}{1em}{}
	
%\addbibresource{refs.bib}
%======================================================================

\newcommand{\plotrow}[2]{
	\rotatebox{90}{\textbf{#2}} & 
		\includegraphics[scale=0.32]{./fig/#1-plot} & 
		\includegraphics[scale=0.32]{./fig/#1-Genv} & 
		\includegraphics[scale=0.32]{./fig/#1-Fenv} & 
		\includegraphics[scale=0.32]{./fig/#1-Kenv} \\
}

%======================================================================

\begin{document}

%------------------------------------------------------------------
\begin{landscape} % spatial distribution plots

\begin{figure}[!ht]	% spatial distribution plots 16-04-30
\centering
\caption{Spatial distribution for images acquired on 14-10-09}

\begin{footnotesize}
	\begin{tabular}{M{1cm}M{5cm}M{5cm}M{5cm}M{5cm}}
	 & \textbf{Point pattern} & \textbf{G-function} & \textbf{F-function} & \textbf{K-function} \\ 
	 \plotrow{incl-l-141009}{All bad pixels} 
	 \plotrow{excl-l-141009}{Excl. locally bright/dim} 
	 \plotrow{only-l-141009}{Only locally bright/dim} 
	\end{tabular}
\end{footnotesize}

\end{figure}

\begin{figure}[!ht]	% spatial distribution plots 16-04-30
\centering
\caption{Spatial distribution for images acquired on 16-04-30}
\begin{footnotesize}
	\begin{tabular}{M{1cm}M{5cm}M{5cm}M{5cm}M{5cm}}
	 & \textbf{Point pattern} & \textbf{G-function} & \textbf{F-function} & \textbf{K-function} \\
	 \plotrow{incl-l-160430}{All bad pixels} 
	 \plotrow{excl-l-160430}{Excl. locally bright/dim}
	 \plotrow{only-l-160430}{Only locally bright/dim} 
	\end{tabular}
\end{footnotesize}


\end{figure}

\end{landscape}
%------------------------------------------------------------------


\cb{Quadrat tests were run over each data set shown above - in all cases, CSR was emphatically rejected, both with and without locally bright pixels, and when quadrats were based on subpanels or half-subpanels (quarter-subpanels weren't attempted because expected number of observations per quadrat was so small)}

\cb{Spatial distribution is dominated by locally bright pixels because they are so much more numerous - splitting between locally and globally bright pixels, in particular the G-function of the globally bright pixels is quite different in shape. \nb{Will fit parametric model to inhomogeneous intensity with and without locally bright pixels and compare the resulting parameters - expect shape of distribution (not necessarily scale) to be same in both cases}}

\cb{G-function plots show an interesting variation in shape at the left-hand end. Suspect that large vertical step indicates adjacent pixels (not the result of accidentally leaving the lines in - that takes the vertical line up to around 0.8!). When locally bright pixels are included as well, there is a short horizontal step before the vertical step; this part of the distribution is presumably dominated by the large numbers of randomly-distributed locally bright pixels, which don't tend to appear as part of `bleeds' or clusters of adjacent pixels. Interestingly, the vertical line is longer in the older images (both with and without locally bright pixels), because a higher proportion of the bad pixels are globally bright pixels that form clusters - the number of locally bright pixels is much higher in the later images.}

\hrulefill

\section*{Next steps}

\todo{Model inhomogeneous intensity - parametrically or via kernel smoothing?}

\todo{Now that the code is in place to produce all of these plots, we can easily carry out a sensitivity analysis to check effects of changes to thresholds \& classification methods}

\end{document}
