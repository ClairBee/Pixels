\documentclass[10pt,fleqn]{article}
\usepackage{/home/clair/Documents/mystyle}

%%%%%%%%%%%%%%%%%%%%%%%%%%%%%%%%%%%%%%%%%%%%%%%%%%%%%%%%%%%%%%%%%%%%%%%%%%%%%%%%%%%%%%%%%%%%%%%%%%%%%%%%%%%%

% Potentially more rigourous approach: use 'top hat' method across slices of grey scale.
% Threshold (arbitrarily) at the mean. Use closing to find shapes of this size.
% Threshold (arbitrarily) just below mean. Use closing to find shapes of this size.

% Could categorise spots according to the layer in which they disappear














%%%%%%%%%%%%%%%%%%%%%%%%%%%%%%%%%%%%%%%%%%%%%%%%%%%%%%%%%%%%%%%%%%%%%%%%%%%%%%%%%%%%%%%%%%%%%%%%%%%%%%%%%%%%

%----------------------------------------------------------------------
% reformat section headers to be smaller \& left-aligned
\titleformat{\section}
	{\normalfont\bfseries}
	{\thesection}{1em}{}
	
\titleformat{\subsection}
	{\normalfont\bfseries}
	{\llap{\parbox{1cm}{\thesubsection}}}{0em}{}
	
%----------------------------------------------------------------------
% SPECIFY BIBLIOGRAPHY FILE & FIELDS TO EXCLUDE
\addbibresource{refs.bib}
%\AtEveryBibitem{\clearfield{url}}
%\AtEveryBibitem{\clearfield{doi}}
%\AtEveryBibitem{\clearfield{isbn}}
%\AtEveryBibitem{\clearfield{issn}}

%----------------------------------------------------------------------
% define code format
\lstnewenvironment{code}[1][R]{			% default language is R
	\lstset{language = #1,
        		columns=fixed,
    			tabsize = 4,
    			breakatwhitespace = true,
			showspaces = false,
    			xleftmargin = .75cm,
    			lineskip = {0pt},
			showstringspaces = false,
			extendedchars = true
    }}
    {}
    
%======================================================================

\begin{document}


\section*{Identification of spots on beryllium screen}

As the electron beam interacts with the tungsten target, the tungsten is heated up. Occasionally flecks of molten tungsten ping off and attach themselves to the beryllium window, where they will remain until the window is replaced.

These flecks manifest in the white and grey images as patches of dim pixels, where the x-ray source is partially blocked. While these do not indicate a problem with the detector panel itself, their locations need to be identified so that the artificially low values can be taken into consideration, for a number of reasons:

\begin{itemize}
\item Identification of affected pixels allows us to quantify the damage to the beryllium window, helping the operator to decide when it should be replaced (in terms of proportion of window affected, areas of window affected, and whether flat-field correction is able to fix the issue)

\item Particularly large spots may still be visible after flat-field image correction; if the locations are known, these can be double-checked and added to a bad pixel map.

\item The presence of a sufficiently large proportion of dim pixels may affect the fit of the distributions used to calculate thresholds for bright and dim pixels. Removing them before fitting the distribution should lead to better threshold setting. (Although actually, in the images checked so far, dim patches along the upper and lower edges of the panel have more of an effect)
\end{itemize}

%\todo{summarise per spot: height, width, n. cells, ?min.thickness?, is spot still visible after flat field correction? There may be a size above which the spot is too dense to be corrected (although this implies that spot volume is linearly correlated with spot area, which it may not be: maybe broader spots were travelling faster when they hit, so spread further?)}

\FloatBarrier
\subsection*{Procedure for identifying spots in image}
Our current aim is to identify large clusters of pixels that are lower in value than the pixels in the surrounding area. Because the specks on the screen are likely to be thinner at their outer edge than in their centre, we expect to see a smooth gradient across these shapes, rather than a clearly defined boundary, so can't simply look for sharp changes in gradient across the image. Instead, we look for brief dips in pixel value.

\begin{description}		% description of stages of identification

\item[Smooth] If the detector were perfect and undamaged, we would expect the values to vary smoothly from one edge of the panel to another. Since the shape of that smooth image is of no interest to us, we can use Loess smoothing to obtain an approximation to the ideal curve along each column, and subtract the smoothed values from the original pixel values to obtain a residual with mean approximately 0 and much lower variance than the original data \textit{(Generally a smoothing span of 0.2 gives good results, but need to do some more investigation into parameter setting)}

\item[Flatten] We're currently only interested in dim values that are lower than their neighbours, so we can homogenize the pixels that we know to be bright; for any pixel whose residual value is greater than the mean, set that pixel's value to the mean.

\item[Closing] We perform a morphological closing over the remaining features, first dilating and then eroding them. The structuring element used is a disc of diameter 5 pixels; a disc, because we can reasonably expect the blobs to have rounded edges, and 5 pixel diameter because each pixel is approximately 0.2mm wide, so this corresponds to a speck of 1mm diameter. Any patches of dim pixels that are unable to contain the structuring element will disappear after the dilation is performed. 

\item[Threshold] The R function \texttt{mmand::threshold}, with thresholding method \texttt{`kmeans'}, automatically determines a cutoff point and classifies each pixel as either light or dark.

\end{description}

\begin{figure}[!ht]		% show stages of identification
\caption{Stages in identification of spots on beryllium screen}
\centering

\captionsetup[subfigure]{position=t}

\begin{subfigure}[b]{0.32\textwidth}
\caption{Raw data: pixelwise means from white image}
\includegraphics[scale=0.32]{./fig/white-image}
\end{subfigure}
%
\begin{subfigure}[b]{0.32\textwidth}
\caption{Loess smoothing along columns with span 1/5}
\includegraphics[scale=0.32]{./fig/loess-smoothing}
\end{subfigure}
%
\begin{subfigure}[b]{0.32\textwidth}
\caption{Residuals after Loess smoothing \\}
\includegraphics[scale=0.32]{./fig/loess-residuals}
\end{subfigure}
%
\par\bigskip
%
\begin{subfigure}[b]{0.32\textwidth}
\caption{Flatten image: brighter-than-average residuals are set to the mean value}
\includegraphics[scale=0.32]{./fig/loess-truncated}
\end{subfigure}
%
\begin{subfigure}[b]{0.32\textwidth}
\caption{Closing: small dark shapes are filtered out \\}
\includegraphics[scale=0.32]{./fig/dilated}
\end{subfigure}
%
\begin{subfigure}[b]{0.32\textwidth}
\caption{Threshold: pixels are classified using a $k$-means algorithm \\}
\includegraphics[scale=0.32]{./fig/thresholded}
\end{subfigure}
\end{figure}

\FloatBarrier
\subsubsection*{A brief overview of the morphological operations used}
A morphological closing consists of two translations of the data: given a set of points $X \in \mathbb{R}^2$ and a structuring element $B$ \cite{Vincent1997},

\begin{description}		% definition of erosion, dilation, closing
\item[Dilation] $\delta_B(X)$ of $X$ by $B$ is the set of points $x \in R^2$ such that the translation of $B$ by $x$ has a non-empty intersection with set $X$.
\item[Erosion] $\varepsilon_B(X)$ of $X$ by $B$ is the set of points $x \in R^2$ such that the translation of $B$ by $x$ is included in $X$.
\item[Closing] of $X$ by $B$ is given by $\phi_B(X) = \varepsilon_B(\delta_B(X))$.
\end{description}

In a greyscale image such as ours, the effect of a dilation is to dilate the brighter parts of the image, shrinking the darker parts. An erosion will erode the brighter parts of the image, expanding the darker parts. A closing, which consists of a dilation followed by an erosion, has the effect of filtering out any dark spots smaller than the structuring element used (in this case, a circle of diameter 5 pixels), while smoothing the edges of any larger features.

Useful examples of the operations can be found in the documentation for R package \texttt{mmand} at \url{https://cran.r-project.org/web/packages/mmand/README.html}

\begin{figure}[ht!]
\centering
\caption{Closing of features of differing complexity, showing changes to feature boundary after closing with structuring element $B$; the light blue hatched area is the new boundary at each step.}
%
\begin{subfigure}[b]{0.48\textwidth}
\caption{Simple (convex) feature}
\centering
\includegraphics[scale=0.18]{./fig/cl-simple-1-org.pdf}
\includegraphics[scale=0.18]{./fig/cl-simple-2-dilated.pdf}
\includegraphics[scale=0.18]{./fig/cl-simple-3-eroded.pdf}
\end{subfigure}
%
\begin{subfigure}[b]{0.48\textwidth}
\caption{Complex (concave) feature}
\centering
\includegraphics[scale=0.18]{./fig/cl-complex-1-org.pdf}
\includegraphics[scale=0.18]{./fig/cl-complex-2-dilated.pdf}
\includegraphics[scale=0.18]{./fig/cl-complex-3-eroded.pdf}
\end{subfigure}
\end{figure}


\FloatBarrier
\subsection*{Damage progression: spots appearing in each acquisition}
Tracking the development of spots automatically is tricky because the screen is able to move slightly in relation to the detector panel between acquisition dates, so successive sets of spots are slightly offset from one another. Occasionally the smaller spots are not detected, but the larger spots identified seem to be consistent among successive acquisitions. 

Presumably the screen was cleaned or replaced between January 26th and May 29th 2015, since the spots evident before January 26th disappear after this acquisition, with no new spots appearing until August 28th 2015 (pending confirmation of this from Jay). Interestingly, spots don't seem to appear gradually over consecutive acquisitions, but suddenly in a single burst, suggesting that this only occurs in a small number of uses.

\begin{table}[!ht]	% table of spot counts & coverage
\caption{Number of spots identified, and proportion of image affected, in successive acquisitions. In some images, smaller spots are not identified correctly, but larger spots are consistently picked up.}
\centering
\begin{footnotesize}
\csvreader[tabular=l|ccc|c,
    		   table head= \textbf{Date} & \textbf{No. spots} & \textbf{Pixels covered} & \textbf{Proportion covered} & \textbf{In plot} \\\hline]%
{./fig/spot-summary.csv}{dt=\dt, nspots=\nspots, px=\px, prop=\prop, legend=\legend}%
{\dt & \nspots &\px & \prop \%  & \legend}%
\end{footnotesize}
\end{table}

\begin{figure}[!ht]	% map spot progression
\centering
\caption{Progression of spots on beryllium screen: each colour represents a single acquisition, as shown in the table above.}
%
\begin{subfigure}[b]{0.3\textwidth}
\caption{All spots}
\centering
\includegraphics[scale=0.3]{./fig/All-screen-spots}
\end{subfigure}
%
\begin{subfigure}[b]{0.3\textwidth}
\caption{Spots prior to Jan-2015}
\centering
\includegraphics[scale=0.3]{./fig/Screen-spots-pre-Jan}
\end{subfigure}
%
\begin{subfigure}[b]{0.3\textwidth}
\caption{Spots after Jan-2015}
\centering
\includegraphics[scale=0.3]{./fig/Screen-spots-post-Jan}
\end{subfigure}

\end{figure}

\FloatBarrier
\subsection*{Fitting distributions to grey/white images before and after spot identification \& edge removal}
Removal of dim spots before fitting a Johnson distribution doesn't make a huge amount of difference to the fit of the distribution. A bigger improvement in the fit of the model (in the dimmer part of the support especially) is obtained by cropping the edge pixels. After cropping the image and removing dim patches caused by spots on the screen, only a very small number of dim pixels remain.

\begin{figure}[!ht]
\caption{Histograms of residuals after fitting a parametric model (circular spot, linear gradient in each panel) to the white images. Cropping out the edge pixels (in this case, 10px at each edge of the image) gives an improved fit, particularly to the left of the body of the distribution.
\\
The Johnson distribution fitted to each histogram is shown as a red line, with potential thresholds at the 0.001- and 99.9-quantiles marked as orange lines (shaded blue for dimmer pixels, yellow for brighter) and the 0.0001- and 99.99-quantiles marked as turquoise lines (shaded green for dimmer pixels, orange for brighter).}
\centering
%
\begin{subfigure}[b]{0.3\textwidth}
\caption{All residuals}
\centering
\includegraphics[scale=0.3]{./fig/res-all}
\end{subfigure}
%
\begin{subfigure}[b]{0.3\textwidth}
\caption{Dim spots removed}
\centering
\includegraphics[scale=0.3]{./fig/res-without-spots}
\end{subfigure}
%
\begin{subfigure}[b]{0.3\textwidth}
\caption{Dim spots removed, edges cropped}
\centering
\includegraphics[scale=0.3]{./fig/res-cropped-without-spots}
\end{subfigure}

\end{figure}

\FloatBarrier
\hrulefill
\printbibliography
\end{document}
