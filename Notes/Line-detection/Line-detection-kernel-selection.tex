\documentclass[10pt,fleqn]{article}
\usepackage{/home/clair/Documents/mystyle}

%================================================================================================================

\newcommand{\UpperCol}{427}
\newcommand{\UpperYrng}{1100:1300}


\newcommand{\LowerCol}{809}
\newcommand{\LowerYrng}{50:200}

%================================================================================================================
\begin{document}

\nb{After thresholding, what is the minimum value \& line length likely to be picked up by this method?}

\section*{Detection of bright/dim lines}

Two sections of the images acquired on March 14th 2016 have already been identified as containing lines of slightly bright pixels. Candidate methods will first be tested on these regions of the black image, to assess their affectiveness.

Ideally, the selected method will isolate single columns of pixels that are brighter than their neighbours (the chosen kernel can be transposed or inverted to identify rows and dimmer lines respectively), and will be able to distinguish genuinely bright lines from the edges of subpanels.

\begin{figure}[!ht]
\caption{Subsets of images used to compare and develop edge detection methods. The columns of interest are plotted in black, with neighbouring columns plotted in light blue and gold, and the next neighouring columns in orange and green. The median value of the bright line segment is shown in red, that of the normal line segment in green.\\ In both cases, the distance between the healthy and unhealthy pixels is aproximately 300 grey values.}
\centering

\begin{footnotesize}
%
\begin{subfigure}[b]{0.22\textwidth}
\caption{Column \UpperCol, upper panel (rows \UpperYrng)}
	\includegraphics[scale=0.2]{./fig/im-raw-upper}
\end{subfigure}
%
\hspace*{\fill}
%
\begin{subfigure}[b]{0.22\textwidth}
\caption{Transect along column \UpperCol}
	\includegraphics[scale=0.2]{./fig/trans-raw-upper}
\end{subfigure}
%
\hspace*{\fill}
%
\begin{subfigure}[b]{0.22\textwidth}
\caption{Column \LowerCol, lower panel (rows \LowerYrng)}
	\includegraphics[scale=0.2]{./fig/im-raw-lower}
\end{subfigure}
%
\hspace*{\fill}
%
\begin{subfigure}[b]{0.22\textwidth}
\caption{Transect along column \LowerCol}
	\includegraphics[scale=0.2]{./fig/trans-raw-lower}
\end{subfigure}
\end{footnotesize}
\end{figure}

\section{Identifying candidate pixels}

The image is convolved with a series of filters designed to enhance vertical edge features within the data (the same filters can also be rotated to identify horizontal edge features):

\begin{figure}[!ht]	% filter examples
\caption{Edge filters applied to the raw image}
\centering
\label{fig:filters}
\begin{footnotesize}

%
\begin{subfigure}[t]{0.15\textwidth}
\caption{Linear,\\ vertical}
$ \begin{matrix} 1 \\ 1 \\ 1 \end{matrix} $
\end{subfigure}
%
\begin{subfigure}[t]{0.15\textwidth}
\caption{Linear,\\ horizontal}
$ \begin{matrix} -1 & 2 & -1 \end{matrix} $
\end{subfigure}
%
\begin{subfigure}[t]{0.15\textwidth}
\caption{3x3 \\ square}
$ \begin{matrix} -1 & 2 & -1 \\ -1 & 2 & -1 \\ -1 & 2 & -1 \end{matrix} $
\end{subfigure}
%
\begin{subfigure}[t]{0.2\textwidth}
\caption{5x5 \\ square}
$ \begin{matrix} -1 & -1 & 4 & -1 & -1 \\ -1 & -1 & 4 & -1 & -1 \\ -1 & -1 & 4 & -1 & -1 \\ -1 & -1 & 4 & -1 & -1 \\ -1 & -1 & 4 & -1 & -1 \end{matrix} $
\end{subfigure}
%
\begin{subfigure}[t]{0.15\textwidth}
\caption{Sobel filter\\}
$ \begin{matrix} -1 & 0 & 1 \\ -2 & 0 & 2 \\ -1 & 0 & 1 \end{matrix} $
\end{subfigure}
%
\begin{subfigure}[t]{0.15\textwidth}
\caption{Laplacian filter\\}
$ \begin{matrix} 0 & -1 & 0 \\ -1 & 4 & -1 \\ 0 & -1 & 0 \end{matrix}$
\end{subfigure}
%\end{raggedleft}
%
\end{footnotesize}
\end{figure}

\FloatBarrier
%-----------------------------------------------------------------------------------
\subsection{Focal filter (linear 1, 1, 1 kernel)}

\begin{figure}[!ht]
\caption{The image is convolved with a vertical linear kernel $(1, 1, 1)$ to highlight vertical sequences of high values.}
\centering
%
\begin{subfigure}[b]{0.22\textwidth}
\caption{Upper panel}
\includegraphics[scale=0.2]{./fig/conv-lin-solid-horiz-upper-im}
\end{subfigure}
%
\begin{subfigure}[b]{0.22\textwidth}
\caption{Upper panel transect}
\includegraphics[scale=0.2]{./fig/conv-lin-solid-horiz-upper-trans}
\end{subfigure}
%
\begin{subfigure}[b]{0.22\textwidth}
\caption{Lower panel}
\includegraphics[scale=0.2]{./fig/conv-lin-solid-horiz-lower-im}
\end{subfigure}
%
\begin{subfigure}[b]{0.22\textwidth}
\caption{Lower panel transect}
\includegraphics[scale=0.2]{./fig/conv-lin-solid-horiz-lower-trans}
\end{subfigure}
%
\end{figure}

\FloatBarrier
%-----------------------------------------------------------------------------------
\subsection{Focal filter (linear -1, 2, -1 kernel)}
\label{sec:linear121}

\begin{figure}[!ht]
\caption{The image is convolved with a horizontal linear kernel $(-1, 2, -1)$ to enhance separation between high points and their horizontal neighours. \\Dashed lines show the median and the median $\pm$ multiples of the MAD of the convolved image.}
\centering
%
\begin{subfigure}[b]{0.22\textwidth}
\caption{Upper panel}
\includegraphics[scale=0.2]{./fig/conv-lin-horiz-upper-im}
\end{subfigure}
%
\begin{subfigure}[b]{0.22\textwidth}
\caption{Upper panel transect}
\includegraphics[scale=0.2]{./fig/conv-lin-horiz-upper-trans}
\end{subfigure}
%
\begin{subfigure}[b]{0.22\textwidth}
\caption{Lower panel}
\includegraphics[scale=0.2]{./fig/conv-lin-horiz-lower-im}
\end{subfigure}
%
\begin{subfigure}[b]{0.22\textwidth}
\caption{Lower panel transect}
\includegraphics[scale=0.2]{./fig/conv-lin-horiz-lower-trans}
\end{subfigure}
%
\end{figure}

\FloatBarrier
%-----------------------------------------------------------------------------------
\subsection{Focal filter (3-square kernel)}

\begin{figure}[!ht]
\caption{The image is convolved with a $3\times 3$ vertical square kernel where each row is identical to the horizontal linear kernel used in Section~\ref{sec:linear121}; this will highlight vertical sequences of pixels that are higher than their hoizontal neighbours.\\ The median, and the median $\pm$ multiples of the MAD are marked with dashed lines.}
\centering
%
\begin{subfigure}[b]{0.22\textwidth}
\caption{Upper panel}
\includegraphics[scale=0.2]{./fig/conv-sq-horiz-upper-im}
\end{subfigure}
%
\begin{subfigure}[b]{0.22\textwidth}
\caption{Upper panel transect}
\includegraphics[scale=0.2]{./fig/conv-sq-horiz-upper-trans}
\end{subfigure}
%
\begin{subfigure}[b]{0.22\textwidth}
\caption{Lower panel}
\includegraphics[scale=0.2]{./fig/conv-sq-horiz-lower-im}
\end{subfigure}
%
\begin{subfigure}[b]{0.22\textwidth}
\caption{Lower panel transect}
\includegraphics[scale=0.2]{./fig/conv-sq-horiz-lower-trans}
\end{subfigure}
%
\end{figure}

\FloatBarrier
%-----------------------------------------------------------------------------------
\subsection{Focal filter (5-square kernel)}

\begin{figure}[!ht]
\caption{The image is convolved with a $5\times 5$ vertical square kernel, with identical rows (-1, -1, 4, -1, -1); this will highlight larger vertical sequences of pixels that are higher than their hoizontal neighbours. \\ The median, and the median $\pm$ multiples of the MAD are marked with dashed lines.}
\centering
%
\begin{subfigure}[b]{0.22\textwidth}
\caption{Upper panel}
\includegraphics[scale=0.2]{./fig/conv-sq-big-horiz-upper-im}
\end{subfigure}
%
\begin{subfigure}[b]{0.22\textwidth}
\caption{Upper panel transect}
\includegraphics[scale=0.2]{./fig/conv-sq-big-horiz-upper-trans}
\end{subfigure}
%
\begin{subfigure}[b]{0.22\textwidth}
\caption{Lower panel}
\includegraphics[scale=0.2]{./fig/conv-sq-big-horiz-lower-im}
\end{subfigure}
%
\begin{subfigure}[b]{0.22\textwidth}
\caption{Lower panel transect}
\includegraphics[scale=0.2]{./fig/conv-sq-big-horiz-lower-trans}
\end{subfigure}
%
\end{figure}

\FloatBarrier
%-----------------------------------------------------------------------------------
\subsection{Sobel filter}

\begin{figure}[!ht]
\caption{The image is convolved with a $3\times 3$ Sobel kernel, commonly used in edge detection.\\ The median, and the median $\pm$ multiples of the MAD are marked with dashed lines.}
\centering
%
\begin{subfigure}[b]{0.22\textwidth}
\caption{Upper panel}
\includegraphics[scale=0.2]{./fig/conv-sobel-upper-im}
\end{subfigure}
%
\begin{subfigure}[b]{0.22\textwidth}
\caption{Upper panel transect}
\includegraphics[scale=0.2]{./fig/conv-sobel-upper-trans}
\end{subfigure}
%
\begin{subfigure}[b]{0.22\textwidth}
\caption{Lower panel}
\includegraphics[scale=0.2]{./fig/conv-sobel-lower-im}
\end{subfigure}
%
\begin{subfigure}[b]{0.22\textwidth}
\caption{Lower panel transect}
\includegraphics[scale=0.2]{./fig/conv-sobel-lower-trans}
\end{subfigure}
%
\end{figure}

Unlike the other filters, the Sobel filter does not highlight the edge pixels, instead highlighting the adjacent columns. Since this would require extra processing to extract the rows of real interest, we can discard this filter as a potential candidate. 

\FloatBarrier
%-----------------------------------------------------------------------------------
\subsection{Laplace filter}

\begin{figure}[!ht]
\caption{The image is convolved with a $3\times 3$ Laplacian kernel, commonly used in edge detection.  \\ The median, and the median $\pm$ multiples of the MAD are marked with dashed lines.}
\centering
%
\begin{subfigure}[b]{0.22\textwidth}
\caption{Upper panel}
\includegraphics[scale=0.2]{./fig/conv-laplace-upper-im}
\end{subfigure}
%
\begin{subfigure}[b]{0.22\textwidth}
\caption{Upper panel transect}
\includegraphics[scale=0.2]{./fig/conv-laplace-upper-trans}
\end{subfigure}
%
\begin{subfigure}[b]{0.22\textwidth}
\caption{Lower panel}
\includegraphics[scale=0.2]{./fig/conv-laplace-lower-im}
\end{subfigure}
%
\begin{subfigure}[b]{0.22\textwidth}
\caption{Lower panel transect}
\includegraphics[scale=0.2]{./fig/conv-laplace-lower-trans}
\end{subfigure}
%
\end{figure}

\FloatBarrier

%================================================================================================================

\section{Comparison of results for each kernel}

\begin{table}[!ht]	% median, separation for each kernel
\caption{Results of convolution of black image acquired on 16-03-14 with various kernels. \\
The median value is calculated for the whole image, and also for the subset of pixels identified as bad, and those that lie on the brighter edge of the 32 subpanels. \\ 
`Bad px separation' is the difference between the median of the bad pixel values and the median of all values, divided by the MAD of all values.\\
`Bad px/edge separation' is the difference between the median of the bad pixel values and the median of the pixels lying on a panel edge, divided by the MAD of all values.\\}
\begin{footnotesize}
\csvreader[tabular = r|ccc|c|cc,
    		   table head = Kernel & Median & Median (bad lines) & Median (panel edges) & MAD & Bad px separation & Bad px/edge separation \\\hline]%
{./fig/df-conv-black-160314.csv}{1=\img, mean=\mean, mean.bad=\meanB, mean.edge=\meanP, median=\median, median.bad=\medB, median.edge=\medP, mad=\mad, mad.bad=\madB, mad.edge=\madP, sd=\sd, sd.bad=\sdB, sd.edge=\sdP, med.ratio=\medRatio, med.diff.mad=\medSep, med.edge.diff.mad=\edgeSep}%
{\img & \median & \medB & \medP & \mad & \medSep & \edgeSep}%
\end{footnotesize}
\end{table}

The highest degree of separation between the median of the known bright lines and of the healthy pixel population, and between the medians of the known bright lines and the panel edges, is given by the horizontal linear kernel and the two square kernels. The rest can be discarded.

\begin{figure}[!ht]	% histograms showing separation
\caption{Separation of known bright columns (red histogram) from subpanel edges (black histogram) using the three most successful kernels}
\centering

%
\begin{subfigure}[b]{0.32\textwidth}
\caption{Horizontal kernel}
\includegraphics[scale=0.3]{./fig/conv-hist-lin}
\end{subfigure}
%
\begin{subfigure}[b]{0.32\textwidth}
\caption{3x3 square kernel}
\includegraphics[scale=0.3]{./fig/conv-hist-3x3}
\end{subfigure}
%
\begin{subfigure}[b]{0.32\textwidth}
\caption{5x5 square kernel}
\includegraphics[scale=0.3]{./fig/conv-hist-5x5}
\end{subfigure}
%

\end{figure}

The linear kernel gives a bimodal distribution for the bad pixels after convolution, possibly because of the oscillations already observed along columns in the  black images (the effect does not appear in the panel edges, presumably because the jump between panels disrupts the pattern); using a square kernel smooths the effect. The 3x3 and 5x5 square kernels offer a very similar performance; the larger kernel gives slightly better separation, but parsimony suggests that we should prefer the smaller.

\nb{Does 3x3 kernel always give MAD comparable to that of raw image?}

\section*{ }
\end{document}

%%%%%%%%%%%%%%%%%%%%%%%%%%%%%%%%%%%%%%%%%%%%%%%%%%%%%%%%%%%%%%%%%%%%%%%%%%%%%%%%%%%%%%%%%%%%%%%%%%%%%%%%%%%%%%%%%%%%%%%%%%%%%%%%%%%%%%%%%%%%%%%%%%%%%%%%%%%%%%%%%
%%%%%%%%%%%%%%%%%%%%%%%%%%%%%%%%%%%%%%%%%%%%%%%%%%%%%%%%%%%%%%%%%%%%%%%%%%%%%%%%%%%%%%%%%%%%%%%%%%%%%%%%%%%%%%%%%%%%%%%%%%%%%%%%%%%%%%%%%%%%%%%%%%%%%%%%%%%%%%%%%
\section{Identifying lines}

\nb{Update this depending on most effective method for identifying streaks of bright/dim pixels above}


Pixels are first assigned a classification according to how much brighter they are than their horizontal neighbours. The MAD of the convolved image is used as a natural unit, with points assigned a score according to how many integer multiples of the MAD they are above the median.\\
While the panel edges are still visible, they have a much lower value after convolution (typically between 3 and 4 MAD above the median) than the bright lines in which we are interested


\begin{figure}[!ht]
\caption{Pixel values after convolution with square kernel, showing distance above median, measured in multiples of MAD.\\ The line in the lower panel has a fainter patch approximately between rows 200 and 300, suggesting that some image reconstruction may be necessary.}
\centering
%
%
\begin{subfigure}[c]{0.4\textwidth}
\caption{Upper panel}
\includegraphics[scale=0.35]{./fig/conv-sq-thresholds-upper}
\end{subfigure}
%
\begin{subfigure}[c]{0.1\textwidth}
\caption{Upper panel}
\includegraphics[scale=0.35]{./fig/conv-sq-thresholds-legend}
\end{subfigure}
%
\begin{subfigure}[c]{0.4\textwidth}
\caption{Lower panel}
\includegraphics[scale=0.35]{./fig/conv-sq-thresholds-lower}
\end{subfigure}
%
%
\end{figure}


\todo{Threshold data}
\todo{Clump resulting bright pixels}

\nb{Also need to repair broken lines, eg. those that pass beneath a dim spot on the screen}

%%%%%%%%%%%%%%%%%%%%%%%%%%%%%%%%%%%%%%%%%%%%%%%%%%%%%%%%%%%%%%%%%%%%%%%%%%%%%%%%%%%%%%
\subsection{Thresholding}

	\subsection{Clump adjacent pixels \& using shape to identify}

	\subsection{Use focal (1,1,1) filter to score pixels that are part of line}

	\subsection{Use focal (1,1,1,1,1) filter to score pixels that are part of line}

\section{Check distribution of edge values in all other images (by colour \& by date)}

		\begin{figure}[!ht]
\caption{Comparison of values (after convolution) of pixels at panel edges vs those manually identified as lying on a bright line, classified according to their distance from the median value in multiples of the MAD. \\ Points outlined in black are edge pixels, those without outline are part of a bright line.\\
The distinction between the distributions of edge and line pixels seems to be clearer in the grey images than in the black.}
\centering
%
\begin{subfigure}[b]{0.3\textwidth}
\caption{Black image, linear kernel}
\includegraphics[scale=0.3]{./fig/props-lin-black}
\end{subfigure}
%
\begin{subfigure}[b]{0.3\textwidth}
\caption{Grey image, linear kernel}
\includegraphics[scale=0.3]{./fig/props-lin-grey}
\end{subfigure}
%
\begin{subfigure}[b]{0.3\textwidth}
\caption{Black image, 3x3 kernel}
\includegraphics[scale=0.3]{./fig/props-3x3-grey}
\end{subfigure}
%
\begin{subfigure}[b]{0.3\textwidth}
\caption{Grey image, 3x3 kernel}
\includegraphics[scale=0.3]{./fig/props-3x3-grey}
\end{subfigure}
%
\begin{subfigure}[b]{0.3\textwidth}
\caption{Black image, 5x5 kernel}
\includegraphics[scale=0.3]{./fig/props-5x5-grey}
\end{subfigure}
%
\begin{subfigure}[b]{0.3\textwidth}
\caption{Black image, 5x5 kernel}
\includegraphics[scale=0.3]{./fig/props-5x5-grey}
\end{subfigure}
%
\end{figure}

\section{Apply this method to grey \& white images. Same results?}

\nb{Lines are more difficult to detect in white image, because all pixels become noisier at higher powers. Should be able to use Jay's sequence of images at different powers to support the argument that we can just use the black \& grey images to identify bright lines (essentially, with \& without spot).}

\section{Apply same method to look for dim lines}
\nb{Better to separate these two processes, rather than to combine: a bright line generally has dimmer lines either side of it in the convolved image (a sort of 1d `Mexican Hat Function'), which may be incorrectly identified as dim lines if we start looking at both together. Far better to run the same process with inverted matrix and see what the result is.}

\section{Apply same method to look for horizontal bright/dim lines}

\section{Behaviour of lines over time}
\todo{Difference between line \& neighbours in each image, observed over time}
\todo{If possible, link to development of root supercluster over time as well} 
\todo{Use transect plots to show difference clearly}

\end{document}
