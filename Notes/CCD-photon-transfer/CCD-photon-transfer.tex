\documentclass[10pt,fleqn]{article}
\usepackage{/home/clair/Documents/mystyle}

%----------------------------------------------------------------------
% reformat section headers to be smaller \& left-aligned
\titleformat{\section}
	{\normalfont\bfseries}
	{\thesection}{1em}{}
	
\titleformat{\subsection}
	{\normalfont\bfseries}
	{\thesubsection}{1em}{}
	
%----------------------------------------------------------------------
% SPECIFY BIBLIOGRAPHY FILE & FIELDS TO EXCLUDE
\addbibresource{refs.bib}
%\AtEveryBibitem{\clearfield{url}}
%\AtEveryBibitem{\clearfield{doi}}
%\AtEveryBibitem{\clearfield{isbn}}
%\AtEveryBibitem{\clearfield{issn}}

   
%======================================================================

\begin{document}

\section{CCD operation \cite{janesick1987}}

CCD operation can be divided into four largely independent functions:
\begin{description}
\item{\textbf{Quantum efficiency (QE)}} - the absorption of a certain fraction of incident photons (or particles) and separation of electron-hole (e-h) pairs by the absorbed energy

\item{\textbf{Charge-collection efficiency (CCE)}} - the movement of electrons into potential wells that correspond to individual pixture elements (pixels)

\item{\textbf{Charge-transfer efficiency (CTE)}} - the sequential transfer of each charge packet through the pixel array and serial register to the readout node.

\item{\textbf{Signal measurement}} - the precise measurement of the charge collected within each pixel, limited by the read noise floor of the CCD.

\end{description}

QE is determined using a calibrated photodetector (by exposing the CCD and the calibrated detector to the same input and comparing the response).

CTE performance may be worse at lower signal levels (< 1000 electrons).

\subsection{Photon-transfer technique}
DN (Digital Number) signal output can be converted into fundamental physical units using

\[J = (\eta_E S_v A_1 A_2)^{-1}, \, \, \, \, K = (S_v A_1 A_2)^{-1}\]

where $J$ and $K$ are given in units of interacting photons/DN and e$^-$/DN, respectively. $S_v$ is the output sensitivity of the CCD, $A_1$ is the camera gain, and $A_2$ is the analog-to-digital converter gain. $\eta_E$ is the effective quantum yield (electrons generated, collected, and transferred per interacting photon per pixel).

$J$ and $K$ can be found graphically by plotting a curve (the photon-transfer curve) of rms noise $\delta_N$ as a function of signal S(DN) for a typical 20 x 20 pixel subarray. Figure{~\ref{fig:photon-transfer-curve} shows an example of such a curve. The signal $S$ is proportional to the average number of interacting photons per pixel with the array uniformly illuminated at some level. The noise $\delta_N$ is the standard deviation of those 400 pixels from the mean at that exposure level after pixel-to-pixel nonuniformity has been removed, which is easily accomplished by differencing two frames taken at the same light level. The read noise floor $\delta_R$ represents the intrinsic noise associated with the readout circuitry, and any other noise sources that are independent of the signal level. As the signal is increased, the noise eventually becomes dominated by the shot noise of the signal (so $\delta_R$ becomes negligible) and is characterised by a line. The intersection of the slope on the signal axis represents $K$ when the light is bright enough (above 3000{\AA}) and $J$ when the light source is lower.

Once $K$ is determined, the full-well capacity of the CCD and its read noise floor can be converted to units of electrons by multiplying the measured DN values by $K$. Since $S_V$, $A_1$ and $A_2$ are constant for a given CCD detector, a decrease in $J$ can be directly attributed to an increase in the quantum yield $\eta_E$. $J$ and $K$ can also be determined with improved accuracy by relating the output signal to its variance as
\[J, K = \frac{S(D)}{\delta_N^2(DN) - \delta_R^2(DN)}.\]

Finding $K$ for several different 20x20 pixel subarrays across the sensor, we obtain the results plotted in Figure~\ref{fig:photon-transfer-hist}; areas that are not well behaved will fall outside of the main histogram.

\begin{figure}[!ht]
\caption{Example of a photon-transfer curve with $\eta_E = 1$.}
\centering
%
\begin{subfigure}[t]{0.49\textwidth}
\caption{Photon transfer curve}
\label{fig:photon-transfer-curve}
\includegraphics[scale=0.5]{photon-transfer-curve} 
\end{subfigure}
%
\begin{subfigure}[t]{0.49\textwidth}
\caption{Histogram of $K$ for several sub-arrays}
\label{fig:photon-transfer-hist}
\includegraphics[scale=0.48]{photon-transfer-hist} 
\end{subfigure}
%
\end{figure}

\subsection{Charge collection efficiency}

CTe affects the output signal in a manner that diverts charge from the target pixel to trailing pixels (commonly called deferred charge). Not all of the charge generated by the illumination at a given site is detected at the output amplifier when that pixel is read out. Even in 1987, CCDs often had CTE greater than 0.999995, so this effect could generally be ignored; however, in some applications \emph{(such as the Hubble WFC, presumably)}, this may become a problem.

If charge is generated within the depletion region associated with a given potential well, there is a high probability that the signal charge will be collected in that pixel. If, on the other hand, the charge is generated within the undepleted, neutral bulk, there is a good probability that the charge may spread into surrounding pixels.

The problem of charge diffusion is particularly important in the case of x-ray events. Depending upon where in the pixel the photon is absorbed, this charge can be divided among two or more neighbouring pixels. X-ray photons absorbed within the depletion region of a given pixel are usually seen as `single=pixel' evenst. Photons absorbed below the depletion region, where the electric field is weaker, create a charge cloud that at first thermally diffuses outward until reaching the rapidly changing potential at the edge of each pixel. At that point, the charge cloud may split into two or more packets, which are collected in more than one pixel.  

\subsection{Charge transfer efficiency}

CTE is generally dominated by `spurious potential pocket': the loss of charge during transfer due to improper potential well shape and/or depth beneath the pixel. Those effects, often due to channel width variation, poly-silicn edge lifting, and typical boron lateral diffusion, have been examined elsewhere.

Various techniques exist to identify spurious potential pocket problems. `Pocket pumping' is described in \nb{ref. 17}. CTE will also degrade with decreasing temperature \emph{(possibly also why this is more of an issue on the WCS)}, but not at levels that are likely to be problematic for us.

\subsection{Read noise}
At high signal levels, the total noise is dominated by pixel-to-pixel sensitivity variations within the device (sometimes referred to as fixed-pattern noise). Pixel nonuniformity peaks when the voltages of transfer phases 2 and 3 are approximately equal, a bias condition that should be avoided if long-term staility in pixel-to-pixel sensitivity is required.

In that the CCD exhibits high linearity, the pixel-to-pixel variations that dominate the noise at high signal levels can be accurately removed so that at high illumination levels the noise will be governed by the shot noise of the signal. This is an important characteristic because the signal-to-noise ratio is then limited by the photon statistics in the signal; this is the best that one can achieve with any detector.

At the lowest illumination levels, the device is limited by noise sources intrinsic to the device. This limiting noise is the read noise floor.

\subsubsection{Trapping-state noise}

The uncertainty in the quantity of charge in a well is due to the trapping and slow release of signal charge either by surface states or by bulk states. Generally due to materials defects (?)

\subsubsection{Reset noise}

The uncertainty in the votage to which the output node is reset after the charge in a pixel is read. Can be removed very effectively using correlated double sampling (CDS) \nb{See ref 18}

\subsubsection{Background noise}

May originate from optical or electrical fat zero, dark current, residual image, or luminescence in the device itself.

Dark generation rates may vary significantly.

Residual charge trapped at the Si-SiO$_2$ interface may cause `blooming' up and down the columns.

\section{Photon transfer curves \cite{janesick2007}}

For a subarray of pixels, rms noise is plotted as a function of average signal at different light levels (or exposure times) (Figure~\ref{fig:photon-transfer-curve-2007}. Measurement accuracy is proportional to the square root of the number of pixels sampled. A $20^2$ pixel subarray will exhibit a 7\% SD for the data products produced. 20,000 pixels will yield 1\% accuracy.

\begin{figure}[!ht]
\caption{Example of a photon-transfer curve with $\eta_E = 1$.}
\centering
%
\begin{subfigure}[t]{0.49\textwidth}
\caption{Ideal PTC showing four classical noise regimes}
\label{fig:photon-transfer-curve-2007}
\includegraphics[scale=0.5]{photon-transfer-curve-2007} 
\end{subfigure}
%
\begin{subfigure}[t]{0.49\textwidth}
\caption{Classical PTC set, plotted in DN units}
\label{fig:photon-transfer-curve-2007-2}
\includegraphics[scale=0.55]{photon-transfer-curve-2007-2} 
\end{subfigure}
%
\end{figure}

Usually the exposure time is varied for a PTC sequence, letting the charge integration period and frame readout time remain constant. For fast PTC generation, the exposure time is increased logarithmically to cover the dynamic range more quickly. Light uniformity across the subarray of pixels being sampled needs to be better than 1\% or FPN measurements will be in error; shot noise is insensitive to field flatness.

The average signal level is plotted only after a fixed average electrical offset level is removed from the raw pixel values that were initially stored in the computer:
\[ S(DN) = \frac{\sum_{i=1}^{N} DN_{i_{ADC}}}{N} - S_{ADC_{OFF}}(DN) \]

where $S(DN)$ is the average signal level, $DN_{i_{ADC}}$ is the raw signal value of the $i$th raw video pixel, $N$ is the number of pixels contained in the subarray, and $S_{ADC_{OFF}} (DN)$ is the ADC offset level.

The FPN must be removed to obtain the shot/read noise response. FPN is eliminated by differencing, pixel by pixel, two identical frames taken back-to-back at the same exposure level. The standard deviation of the difference is given by

\[\sigma_\delta(DN) = \left[\frac{\sum_{i=1}^N(DN1_{i_{ADC}} - DN2_{i_{ADC}})^2}{N}\right]^{1/2} \]

where $DN1_{i_{ADC}}$ and $DN2_{i_{ADC}}$ are the raw signal values of the $i$th raw video pixel for the first frame and the second subarrays taken back-to-back. It is important to divide the result by $2^{1/2}$ because random noise increases by this amount: when two identical frames are either subtracted (as in this case) or added, the random noise component of the resultant frame increases
by $2^{1/2}$. Therefore, the true random noise from the differencing process is
\[ \sigma_{read + shot}(DN) = \frac{\sigma_{\delta}}{2^{1/2}} \]

The noise can be broken into components as in Figure~\ref{fig:photon-transfer-curve-2007}

\hrulefill
\printbibliography

\end{document}
