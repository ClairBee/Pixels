\documentclass[10pt,fleqn]{article}
\usepackage{/home/clair/Documents/mystyle}

%----------------------------------------------------------------------
% reformat section headers to be smaller \& left-aligned
\titleformat{\section}
	{\normalfont\bfseries}
	{\thesection}{1em}{}
	
\titleformat{\subsection}
	{\normalfont\bfseries}
	{\llap{\parbox{1cm}{\thesubsection}}}{0em}{}
	
%======================================================================
\newcommand{\statechange}[3]{
\resizebox{\textwidth}{!}{
		\csvreader[tabular=ll|c|ccc|cc|c|cc|cc, head to column names=true,
			   table head = && \multicolumn{11}{c}{Classification on #2} \\ 
			   				\multirow{11}{*}{\rot{90}{Class. on #1}} & & Normal & No response & Dead & Hot & V. bright & Bright & Bright line & Screen spot & Edge	 & 	V. dim & Dim \\\hline ]%
		  {./fig/bpx-change-#1-#2#3.csv}{}%
		{& \csvlinetotablerow}%
}
	}
%======================================================================

\begin{document}

\section{Pixels identified using globally extreme values}

Bad pixels were identified using simple median-based thresholding of the black, grey and white pixelwise mean values, without first fitting a parametric model. Bright lines were identified through convolution with an edge-detecting filter, and dim spots using a morphological closing.

\begin{table}[!ht]
\caption{Mean proportion of pixels moving from one state to the next at each acquisition}

\resizebox{\textwidth}{!}{
\csvreader[tabular=ll|c|ccc|cc|c|cc|cc, head to column names=true,
			   table head = && \multicolumn{11}{c}{New state} \\ 
			   				\multirow{11}{*}{\rot{90}{Initial state}} & & Normal & No response & Dead & Hot & V. bright & Bright & Bright line & Screen spot & Edge	 & 	V. dim & Dim \\\hline ]%
		  {./fig/mean-change.csv}{}%
		{& \csvlinetotablerow}%
}
		

\end{table}

\begin{table}[!ht]
\caption{Mean number of pixels moving from one state to the next at each acquisition}

\resizebox{\textwidth}{!}{
\csvreader[tabular=ll|c|ccc|cc|c|cc|cc, head to column names=true,
			   table head = && \multicolumn{11}{c}{New state} \\ 
			   				\multirow{11}{*}{\rot{90}{Initial state}} & & Normal & No response & Dead & Hot & V. bright & Bright & Bright line & Screen spot & Edge	 & 	V. dim & Dim \\\hline ]%
		  {./fig/mean-px.csv}{}%
		{& \csvlinetotablerow}%
}
		

\end{table}

With the exception of screen spots (which vary because the screen is replaced, not because of any problems with detector pixels), the categories identified are reasonably stable; there is some movement between bright, very bright and hot states (probably because brighter pixels are generally more variable, so perhaps should not be expected to remain in the same category). However, a more sophisticated classification approach, considering the clusters in which pixels occur, may also be useful.

%%%%%%%%%%%%%%%%%%%%%%%%%%%%%%%%%%%%%%%%%%%%%%%%%%%%%%%%%%%%%%%%%%%%%%%%%%%%%%%%%%%%%%%%%%%%%%%%%%%%%%%%%%%%%%
\clearpage
%%%%%%%%%%%%%%%%%%%%%%%%%%%%%%%%%%%%%%%%%%%%%%%%%%%%%%%%%%%%%%%%%%%%%%%%%%%%%%%%%%%%%%%%%%%%%%%%%%%%%%%%%%%%%%

\section{Pixels classified according to their arrangement}

The `globally extreme' bad pixels given above were grouped into clusters or superclusters according to the composition of their neighbouring cells.

\begin{table}[!ht]
\caption{Mean number of pixels moving from one cluster state to the next at each acquisition}

\begin{footnotesize}
\csvreader[tabular=ll|cccc|c, head to column names=true,
			   table head = && \multicolumn{5}{c}{New state} \\ 
			   				\multirow{6}{*}{\rot{90}{Initially}} & & Normal & Singleton & Cluster & Supercluster & Edge/screen spot \\\hline ]%
		  {./fig/sc-mean-change.csv}{}%
		{& \csvlinetotablerow}%
\end{footnotesize}
\end{table}

\begin{table}[!ht]
\caption{Mean proportion of pixels moving from one cluster state to the next at each acquisition}

\begin{footnotesize}
\csvreader[tabular=ll|cccc|c, head to column names=true,
			   table head = && \multicolumn{5}{c}{New state} \\ 
			   				\multirow{6}{*}{\rot{90}{Initially}} & & Normal & Singleton & Cluster & Supercluster & Edge/screen spot \\\hline ]%
		  {./fig/sc-mean-rate.csv}{}%
		{& \csvlinetotablerow}%
\end{footnotesize}
\end{table}

Again, the proportions in each cluster seem quite stable. However, the proportions in the `supercluster' category are boosted by the fact that bright lines are included in this class; if we were to split those into a separate category, we may find that superclusters without associated bright lines are more prone to switching class.

%%%%%%%%%%%%%%%%%%%%%%%%%%%%%%%%%%%%%%%%%%%%%%%%%%%%%%%%%%%%%%%%%%%%%%%%%%%%%%%%%%%%%%%%%%%%%%%%%%%%%%%%%%%%%%
\clearpage
%%%%%%%%%%%%%%%%%%%%%%%%%%%%%%%%%%%%%%%%%%%%%%%%%%%%%%%%%%%%%%%%%%%%%%%%%%%%%%%%%%%%%%%%%%%%%%%%%%%%%%%%%%%%%%

\section{Next step: locally extreme pixels}

So far, this classification scheme only considers globally extreme values: those points that lie so far from the bulk of the data in any image that they can be easily identified. Rather than identifying slightly bright or dim points using the same method, I considered a convolution-based approach, indetifying any points which are brighter than their immediate neighbourhood by some margin. The approach already tested is to use a median-filtering approach similar to that commonly used to correct bad pixels in image processing: the median value of each pixel's immediate neighbourhood is subtracted from the pixel value, and the remaining data thresholded at some level. As an initial threshold, 2x the MAD of the source image was used; the approach has only been applied to the black images in the tables below, where the MAD ranges from 200 to 300 over the 12 acquisitions. 

\begin{table}[!ht]
\caption{Mean proportion of pixels moving from one state to the next at each acquisition, including locally bright or dim pixels}

\resizebox{\textwidth}{!}{\csvreader[tabular=ll|c|ccc|ccc|c|cc|ccc, head to column names=true,
			   table head = && \multicolumn{13}{c}{New state} \\ 
			   				\multirow{13}{*}{\rot{90}{Initial state}} & & Normal & No response & Dead & Hot & V. bright & Bright & Bright line & Locally bright & Screen spot & Edge	 & 	V. dim & Dim & Locally dim \\\hline ]%
		  {./fig/mean-change-local-2-mad.csv}{}%
		{& \csvlinetotablerow}%
		}
\end{table}

\begin{table}[!ht]
\caption{Mean number of pixels moving from one state to the next at each acquisition, including locally bright or dim pixels}

\resizebox{\textwidth}{!}{\csvreader[tabular=ll|c|ccc|ccc|c|cc|ccc, head to column names=true,
			   table head = && \multicolumn{13}{c}{New state} \\ 
			   				\multirow{13}{*}{\rot{90}{Initial state}} & & Normal & No response & Dead & Hot & V. bright & Bright & Bright line & Locally bright & Screen spot & Edge	 & 	V. dim & Dim & Locally dim \\\hline ]%
		  {./fig/mean-px-local-2-mad.csv}{}%
		{& \csvlinetotablerow}%
		}
\end{table}



\end{document}


































%%%%%%%%%%%%%%%%%%%%%%%%%%%%%%%%%%%%%%%%%%%%%%%%%%%%%%%%%%%%%%%%%%%%%%%%%%%%%%%%%%%%%%%%%%%%%%%%%%%%%%%%%%%%%%
%%%%%%%%%%%%%%%%%%%%%%%%%%%%%%%%%%%%%%%%%%%%%%%%%%%%%%%%%%%%%%%%%%%%%%%%%%%%%%%%%%%%%%%%%%%%%%%%%%%%%%%%%%%%%%



\clearpage

\section*{State changes (numbers of pixels)}

\statechange{141009}{141118}{}

\statechange{141118}{141217}{}

\statechange{141217}{150108}{}

\statechange{150108}{150113}{}

\statechange{150113}{150126}{}

\statechange{150126}{150529}{}

\statechange{150529}{150730}{}

\statechange{150730}{150828}{}

\statechange{150828}{151015}{}

\statechange{151015}{160314}{}

\statechange{160314}{160430}{}

\hrulefill

\section*{State changes by proportion}

\statechange{141009}{141118}{-prop}

\statechange{141118}{141217}{-prop}

\statechange{141217}{150108}{-prop}

\statechange{150108}{150113}{-prop}

\statechange{150113}{150126}{-prop}

\statechange{150126}{150529}{-prop}

\statechange{150529}{150730}{-prop}

\statechange{150730}{150828}{-prop}

\statechange{150828}{151015}{-prop}

\statechange{151015}{160314}{-prop}

\statechange{160314}{160430}{-prop}

\end{document}

