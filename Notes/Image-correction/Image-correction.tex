\documentclass[10pt,fleqn]{article}
\usepackage{/home/clair/Documents/mystyle}

%----------------------------------------------------------------------
% reformat section headers to be smaller \& left-aligned
\titleformat{\section}
	{\normalfont\bfseries}
	{\thesection}{1em}{}
	
\titleformat{\subsection}
	{\normalfont\bfseries}
	{\thesubsection}{1em}{}
	
%----------------------------------------------------------------------
% SPECIFY BIBLIOGRAPHY FILE & FIELDS TO EXCLUDE
\addbibresource{refs.bib}
\AtEveryBibitem{\clearfield{url}}
\AtEveryBibitem{\clearfield{note}}
\AtEveryBibitem{\clearfield{doi}}


%----------------------------------------------------------------------
% define code format
\lstnewenvironment{code}[1][R]{			% default language is R
	\lstset{language = #1,
        		columns=fixed,
    			tabsize = 4,
    			breakatwhitespace = true,
			showspaces = false,
    			xleftmargin = .75cm,
    			lineskip = {0pt},
			showstringspaces = false,
			extendedchars = true
    }}
    {}
    
%======================================================================

\begin{document}

\section{Flat-field correction \cite{wiki:FlatFieldCorrection}}

\subsection{Terminology}

\begin{description}
\item[Flat field correction: ]
 technique used to remove artefacts from 2d printed images, which may be caused by variations in pixel-to-pixel sensitivity. A \textbf{flat field} consists of two numbers per pixel: gain and dark current (or dark frame).

\item[Gain: ] How the signal varies as a function of the amount of `light' - will almost always be a linear function, as this is a ratio of the input and output signals.

\item[Flat field ($F$): ] `Blank' image acquired with x-ray beam on.

\item[Dark field ($D$): ] `Blank' image acquired with x-ray beam off.

\end{description}

\subsection{Application}
Given a raw image $R$ dark frame $D$, flat field image $F$, and $m$ the average value of $(F-D)$, we have the corrected image $C$ and gain $G$:

\[ C = \frac{m(R-D)}{(F-D)} = G(R-D) \qquad \qquad \qquad G = \frac{m}{(F-D)}\]

Fixed-pattern noise is a known limiting factor in x-ray CT imaging, but can be removed using a flat field correction. A projection image with sample $P$ can be normalised to 

\[N = \frac{(P-D)}{(F-D)} \]

but this method is reliant on the x-ray source, scintillator response and CCD sensitivity remaining constant and stationary - which is generally not true in practice.

\section{Dynamic flat-field correction \cite{vanN2015}}
Principal component analysis of a set of flat field images is used to compute eigen flat fields, a linear combination of which can be used to individually normalise each x-ray projection.

\section{Noise in CCD sensors \cite{CCD-noise:2009}}

%\newpage
\hrulefill
\printbibliography

\end{document}