\documentclass[8pt]{beamer}
\usepackage{appendixnumberbeamer}

\usetheme{Singapore}
%\usetheme{Madrid}
%\usetheme{default}

\setbeamertemplate{footline}[frame number] 

%======================================================================
\newcommand{\todo}[1]{
	\begin{minipage}{\textwidth}
	\begin{footnotesize}
	\textcolor{purple}{
					\begin{tabular}{p{0.01\textwidth}p{0.95\textwidth}}
						$\bullet$ & #1
					\end{tabular}
					}
	\end{footnotesize}
	\end{minipage}
}

\newcommand{\nb}[1]{\textcolor{red}{\textit{[#1]}}}
%======================================================================

\hypersetup{pdfpagemode=FullScreen}		% pdf will automatically open in presentation mode

\title{The Life \& Death of Pixels}
%\subtitle{Optional Subtitle}

\author{Clair Barnes}

\institute[University of Warwick] % (optional, but mostly needed)
{
  Department of Statistics \& WMG\\
  University of Warwick
}

\date{Steering Group Meeting, September 20th 2016}

\graphicspath{{./fig/}}


%======================================================================


\begin{document}

\begin{frame}
  \titlepage
\end{frame}

\begin{frame}{Outline}

\nb{Still to do: Table of counts of each pixel classification in each image (split out dense areas?}

  \tableofcontents
  % You might wish to add the option [pausesections]
\end{frame}

%-----------------------------------------------------------------------

\section{Identification of abnormal pixels}

\subsection{Thresholding}

\begin{frame}{Abnormal pixel values}{Thresholding}
Seeking a common approach, applicable to all variables of interest. \\
Therefore we need to avoid distributional assumptions, where possible.
	\begin{itemize}
		\item Cannot assume symmetry.
		\item Cannot assume unimodal distribution.				% will show such plots later on
		\item Expect to see a large number of outliers - need robust measure.		% remark on dark lines at LHS of plot
	\end{itemize}
	
	\begin{figure}[!ht]
		\includegraphics[scale=0.3]{hist-asymmetry}
		\includegraphics[scale=0.3]{hist-outliers}
	\end{figure}
\end{frame}

%-----------------------------------------------------------------------

\begin{frame}{Abnormal pixel values}{Thresholding}
	A solution:
	\begin{itemize}
		\item Cut data at modal density, not at median
		\item Use normalised MAD instead of SD							% note formula for MAD
		\item Calculate MAD on each side of modal density independently
	\end{itemize}
	
	\begin{figure}[!ht]
		\includegraphics[scale=0.3]{hist-bounds-160705}
		\includegraphics[scale=0.3]{hist-bounds-131002}
	\end{figure}
	A robust, asymmetric analogue to $6\sigma$ approach
	
\end{frame}

%======================================================================

\subsection{Pixel attributes}

\begin{frame}{Variables of interest}{Pixelwise mean values}
	\begin{itemize}
		\item Each aquisition consists of 60 images:
		\begin{itemize}
			\item 20 with nominal mean value of 85\% of range (white)
			\item 20 at 1/4 of the current required for the white images (grey)
			\item 20 with no x-ray exposure (black)
		\end{itemize}
		\item Take the mean value of each pixel in each set of 20, to obtain black, grey and white mean images. 
		% explain scale here 
	\end{itemize}
	
	\begin{center}
		\includegraphics[scale=0.11]{image-pwm-black}
		\includegraphics[scale=0.11]{image-pwm-grey}
		\includegraphics[scale=0.11]{image-pwm-white}
	\end{center}
	These are our main source of information about the detector's behaviour
\end{frame}

%-----------------------------------------------------------------------

\begin{frame}{Variables of interest}{Pixelwise mean values}
	Black \& grey pixelwise means are thresholded directly to identify \textbf{globally extreme} pixels.
	\vspace{-12pt}
	\begin{center}
		\includegraphics[scale=0.11]{th-hist-black}
		\includegraphics[scale=0.11]{th-hist-grey}
		
		\includegraphics[scale=0.11]{th-hist-scale}
	\end{center}
	
	\textbf{Dark} pixels: values in grey \& white images are typical of black image\\
	\textbf{Hot} pixels: fully saturated, even in black image

\end{frame}
	
%-----------------------------------------------------------------------

\begin{frame}[label = ms-res]{Variables of interest}{Median-smoothed residuals}
	\begin{itemize}
		\item Smooth pixelwise mean image using median filter
		\item Subtract median-smoothed values from original image
		\item Threshold residuals to identify \textbf{locally extreme} pixels		% sketch image of possible local-but-not-global bright/dim px.
	\end{itemize}
	\begin{center}
		\includegraphics[scale=0.11]{image-med-smoothed-black}
		\includegraphics[scale=0.11]{image-med-residuals-black}
		\hyperlink{ms-res-supplemental}{\includegraphics[scale=0.11]{hist-med-residuals-black}}
	\end{center}
	High-valued residuals are often also globally high-valued
	% Comment on unusual shape of histogram - link to oscillation as separate slide?
\end{frame}
 
%-----------------------------------------------------------------------
 
\begin{frame}{Variables of interest}{Linear residuals}
	\begin{itemize}
		\item Pixel response is expected to be linear 		% manual specifies < 1% non-linearity, whatever that means
		\item Regression of grey value on black \& white value generally v. accurate
		\item Threshold residuals to identify \textbf{nonlinear} pixels
	\end{itemize}		

	\begin{center}
		\includegraphics[scale=0.11]{image-linear-fv}
		\includegraphics[scale=0.11]{image-linear-residual}
		\includegraphics[scale=0.11]{hist-linear-residuals}
	\end{center}
\end{frame}
	
%-----------------------------------------------------------------------
	
\begin{frame}{Variables of interest}{Response to spot}
	\begin{itemize}
		\item Subtract black from grey images to get pixelwise response
		\item Response to conical x-ray beam should be well modelled by 2d Gaussian
		\item fit 2d Gaussian with constraint on location				% note formula: spot central, x & y independent
	\end{itemize}
	
	\begin{center}
		\includegraphics[scale=0.11]{gs-fitted-values}
		\includegraphics[scale=0.11]{gs-org-values}			% contours are at same levels as colour contours, so should align 
		\includegraphics[scale=0.11]{gs-residuals}
	\end{center}
	
\end{frame}		
		
\begin{frame}{Variables of interest}{Response to spot}
	\begin{itemize}
		\item If centre is on constraint boundary, compare constrained model to free model
		\begin{itemize}
			\item Improvement in RMSE suggests spot is off-centre
			\item No improvement suggests response is not elliptical
		\end{itemize}
	\end{itemize}			% link to supplemental slide showing difference in residuals when spot is constrained incorrectly
	
	\begin{center}
		\includegraphics[scale=0.11]{gs-central-spot}
		\includegraphics[scale=0.11]{gs-off-centre-spot}			% why does spot centring matter? Not necessarily for residuals, but for getting right nominal value in the first place when selecting power settings.
		\includegraphics[scale=0.11]{gs-non-circular-response}
	\end{center}

\end{frame}
 
%-----------------------------------------------------------------------

% INCLUDE STANDARD DEVIATION AS ONE OF THE VARIABLES?

\subsection{Summary of classifications}

\begin{frame}{Classification} 

	%All variables are thresholded using the approach outlined previously\\
	% Retain classifications to track spatial distribution & development/progression of each pixel type
	% order is prioritised: if bright in black but dark in white, 'dark' wins
	
	\begin{tabular}{p{0.2\textwidth}p{0.78\textwidth}}
		\textbf{Hot} & Pixel value == 65535 in black images\\
		\textbf{Dark} & Pixel value in white image $\leq$ 15000\\
		\textbf{Very bright} & Pixel value between $Q_{0.5}$ of bright pixels \& maximum value\\
		\textbf{Bright} & Pixel value between $Q_{0.25}$ \& $Q_{0.5}$ of bright pixels\\
		\textbf{Slightly bright} & Pixel value between upper threshold \& $Q_{0.25}$ of bright pixels\\
		\textbf{Dim} & Pixel value below lower threshold\\
		\textbf{Locally bright} & Median-smoothed residual above upper threshold\\
		\textbf{Locally dim} & Median-smoothed residual below lower threshold\\
		\textbf{Nonlinear} & Linear residual outside of thresholds\\
		\textbf{Spot response} & Gaussian spot residual outside of thresholds
	\end{tabular}

\end{frame}
% \todo{table of classifications. Give counts of pixels observed?}



\section{Multi-pixel features}

\subsection{Spots on beryllium window}

\begin{frame}{Spots on beryllium window}
	\begin{itemize}
		\item Visible in some illuminated images as irregular dim patches
		\item Caused by flecks of superheated tungsten sticking to beryllium screen
		\item Can only be removed by replacing the beryllium window
		\item Not always removed by shading correction
		\item Apparent position likely to move between acquisitions	% so simply adding position to bad pixel map won't help
	\end{itemize}
	
	\begin{center}
		\includegraphics[scale=0.11]{image-white-spots}
		\includegraphics[scale=0.11]{image-sc-spots}
		\includegraphics[scale=0.11]{plot-spot-movement}
	\end{center}
\end{frame}

%-----------------------------------------------------------------------

\begin{frame}{Screen spots in linear residuals}

	\begin{itemize}
		\item Spots may also appear in linear residuals
		\item Bright `shadow' indicates different position in grey \& white images
	\begin{center}
		\includegraphics[scale=0.14]{image-l-res-spots}
		\includegraphics[scale=0.14]{plot-l-res-spots}
	\end{center}
		\item May indicate a pit in the tungsten target?
	\end{itemize}	
\end{frame}

%-----------------------------------------------------------------------

\begin{frame}{Identification of screen spots}
% include discussion of structuring element used
	\begin{center}
		\includegraphics[scale=0.1]{finding-spots-raw}
		\includegraphics[scale=0.1]{finding-spots-midline-adj}
		\includegraphics[scale=0.1]{finding-spots-lowess-res}
		\includegraphics[scale=0.1]{finding-spots-truncated-res}
		
		\includegraphics[scale=0.1]{finding-spots-closing}
		\includegraphics[scale=0.1]{finding-spots-thresholded}
		\includegraphics[scale=0.1]{finding-spots-dilated}
		\includegraphics[scale=0.1]{finding-spots-final}
	\end{center}
\end{frame}


%======================================================================

\subsection{! Dense regions}

\begin{frame}{Dense regions of defects} % figs needed
	\begin{itemize}
		\item Dense regions suggest a larger issue
		\item Tend to arise around edges of panels
		\item Can be identified by density filter
	\end{itemize}
	
	\todo{examples of dense regions: 160430 corners, loan panel}
	
\end{frame}

%-----------------------------------------------------------------------

\subsection{! Linear defects}

\begin{frame}{Column defects} % figs needed

\end{frame}

%-----------------------------------------------------------------------

\begin{frame}{Row defects} % figs needed
	\begin{itemize}
		\item Two row defects observed in current loan panel
		\item Row 77: decreasing gradient R-L
			\todo{three transects as before}
		\hrule
		\item Row 1025: constant offset
			\todo{three transects as before}
		\item Assume that these must occur after data is transferred along columns, in readout register?
	\end{itemize}
\end{frame}

%-----------------------------------------------------------------------

\subsection{! Cluster defects}

\begin{frame}{Cluster defects} % figs needed

	
\end{frame}

\section{! Two case studies}

\subsection{! Images from 14-10-09 (first images after refurbishment)}

\begin{frame}{Images from 14-10-09}{Pixel values} % figs needed
	\todo{black pw.m + hist + px identified}
	\todo{grey pw.m + hist + px identified}
\end{frame}

%-----------------------------------------------------------------------

\begin{frame}{Images from 14-10-09}{Median-smoothed residuals} % figs needed
	\todo{black residuals + hist + px identified}
	\todo{grey residuals + hist + px identified}
\end{frame}

%-----------------------------------------------------------------------

\begin{frame}{Images from 14-10-09}{Model residuals} % figs needed
	\todo{linear residuals + hist + px identified}
	\todo{spot residuals + hist + px identified}
\end{frame}
	
%-----------------------------------------------------------------------

\begin{frame}{Images from 14-10-09}{Pixels identified} % figs needed
	\todo{plot of pixels identified (colour-coded \& with counts)}
	\nb{shading-corrected image?}
\end{frame}
	
\subsection{! Images from 16-07-19 (loan panel currently in use)}

\begin{frame}{Images from 16-07-19}{Pixel values} % figs needed
	\todo{black pw.m + hist + px identified}
	\todo{grey pw.m + hist + px identified}
\end{frame}

%-----------------------------------------------------------------------

\begin{frame}{Images from 16-07-19}{Median-smoothed residuals} % figs needed
	\todo{black residuals + hist + px identified}
	\todo{grey residuals + hist + px identified}
\end{frame}

%-----------------------------------------------------------------------

\begin{frame}{Images from 16-07-19}{Model residuals} % figs needed
	\todo{linear residuals + hist + px identified}
	\todo{spot residuals + hist + px identified}
\end{frame}
	
%-----------------------------------------------------------------------

\begin{frame}{Images from 16-07-19}{Pixels identified} % figs needed
	\todo{plot of pixels identified (colour-coded \& with counts)}
\end{frame}


%%%%%%%%%%%%%%%%%%%%%%%%%%%%%%%%%%%%%%%%%%%%%%%%%%%%%%%%%%%%%%%%%%%%%%%%%%%%%%%%%%%%%%%%%%%%%%%%%%%%%%%%%%%%%%%%%%%%%%%%%%%%%%%%%%%%%%%%%%%%%%%%%%%%%%%%%%%
\appendix
%%%%%%%%%%%%%%%%%%%%%%%%%%%%%%%%%%%%%%%%%%%%%%%%%%%%%%%%%%%%%%%%%%%%%%%%%%%%%%%%%%%%%%%%%%%%%%%%%%%%%%%%%%%%%%%%%%%%%%%%%%%%%%%%%%%%%%%%%%%%%%%%%%%%%%%%%%%


\section{Additional: Oscillation in dark images}

\begin{frame}[label = ms-res-supplemental]{Oscillation along columns in dark images}

	`Standing wave' oscillation along columns - not disrupted by bright or dead pixels.
	
	Observed in all images, but not in all columns
	
	\hyperlink{ms-res}{
	\includegraphics[scale=0.3]{osc-transect-130613}	
	\includegraphics[scale=0.3]{osc-transect-141009}	
	\\
	\includegraphics[scale=0.3]{osc-transect-loan}	
	\includegraphics[scale=0.3]{osc-transect-MCT225}	
	}
\end{frame}
%%%%%%%%%%%%%%%%%%%%%%%%%%%%%%%%%%%%%%%%%%%%%%%%%%%%%%%%%%%%%%%%%%%%%%%%%%%%%%%%%%%%%%%%%%%%%%%%%%%%%%%%%%%%%%%%%%%%%%%%%%%%%%%%%%%%%%%%%%%%%%%%%%%%%%%%%%%
\end{document}
%%%%%%%%%%%%%%%%%%%%%%%%%%%%%%%%%%%%%%%%%%%%%%%%%%%%%%%%%%%%%%%%%%%%%%%%%%%%%%%%%%%%%%%%%%%%%%%%%%%%%%%%%%%%%%%%%%%%%%%%%%%%%%%%%%%%%%%%%%%%%%%%%%%%%%%%%%%
