\documentclass[8pt]{beamer}
%\usepackage{nicefrac}

%\usetheme{Singapore}
%\usetheme{Madrid}
\usetheme{default}
%======================================================================
\newcommand{\todo}[1]{
	\begin{minipage}{\textwidth}
	\begin{footnotesize}
	\textcolor{purple}{
					\begin{tabular}{p{0.01\textwidth}p{0.95\textwidth}}
						$\bullet$ & #1
					\end{tabular}
					}
	\end{footnotesize}
	\end{minipage}
}

\newcommand{\nb}[1]{\textcolor{red}{\textit{[#1]}}}
%======================================================================

\hypersetup{pdfpagemode=FullScreen}		% pdf will automatically open in presentation mode

\title{The Life \& Death of Pixels}
%\subtitle{Optional Subtitle}

\author{Clair Barnes}

\institute[University of Warwick] % (optional, but mostly needed)
{
  Department of Statistics \& WMG\\
  University of Warwick
}

\date{Steering Group Meeting, September 20th 2016}


%======================================================================


\begin{document}

\begin{frame}
  \titlepage
\end{frame}

\begin{frame}{Outline}
  \tableofcontents
  % You might wish to add the option [pausesections]
\end{frame}

%-----------------------------------------------------------------------

\section{Identification of defects}

\subsection{Abnormal pixels}

\subsubsection{Thresholding}

\begin{frame}{Abnormal pixel values}{Thresholding}
	\begin{itemize}
		\item Common approach applicable to all variables of interest.
		\item Cannot assume symmetry. \todo{histogram of grey values from `loan' image. Show median}
		\item Expect to see a large number of outliers - need robust measure. \todo{compare SD with MAD on hist}
	\end{itemize}
\end{frame}

%-----------------------------------------------------------------------

\begin{frame}{Abnormal pixel values}{Thresholding}
	A solution:
	\begin{itemize}
		\item Cut data at modal density, not at median
		\item Use NMAD instead of SD
		\item Calculate MAD on each side of modal density independently
			\todo{histograms with thresholds marked}
		\item A robust, asymmetric analogue to $6\sigma$ approach
	\end{itemize}
\end{frame}

%-----------------------------------------------------------------------

\begin{frame}{Abnormal pixel values}{Classification}
	Further classification is straightforward:
	\todo{add limits to histograms}
	\todo{table of classifications}
	
	\nb{Check best approach: split globally extreme px by degree, then remainder by type? if only using degrees on pixelwise mean values, include this in slide on pixelwise mean values instead}
\end{frame}


%======================================================================

\subsubsection{Pixel attributes}

\begin{frame}{Variables of interest}{Pixelwise mean values}
	\begin{itemize}
		\item Each aquisition consists of 60 images:
		\begin{itemize}
			\item 20 with nominal mean value of 85\% of range (white)
			\item 20 at 1/4 of the current required for the white images (grey)
			\item 20 with no x-ray exposure (black)
		\end{itemize}
		\item Take the mean value of each pixel in each set of 20, to obtain black, grey and white mean images. 
			\todo{Insert 14-10-09 black, grey, white images}
		\item These are our main source of information about the detector's behaviour
		\item Black \& grey pixelwise means are thresholded directly to identify \textbf{globally extreme} pixels
	\end{itemize}
\end{frame}

%-----------------------------------------------------------------------

\begin{frame}{Variables of interest}{Median-smoothed residuals}
	\begin{itemize}
		\item Smooth pixelwise mean image using median filter
		\item Subtract median-smoothed values from original image
		\item Threshold residuals to identify \textbf{locally extreme} pixels
		\item High-valued residuals are also globally high-valued
	\end{itemize}
		\todo{smoothed image}
		\todo{residual plot}
		\todo{residual histogram}
\end{frame}
 
%-----------------------------------------------------------------------
 
\begin{frame}{Variables of interest}{Linear residuals}
	\begin{itemize}
		\item Pixel response is expected to be linear 		% manual specifies < 1% non-linearity
		\item Regression of grey value on black \& white value generally v. accurate
		\item Threshold residuals
	\end{itemize}		
		\todo{fitted linear values}
		\todo{Linear residuals}
		\todo{histogram of linear residuals with thresholds}
\end{frame}
	
%-----------------------------------------------------------------------
	
\begin{frame}{Variables of interest}{Response to spot}
	\begin{itemize}
		\item Response to conical x-ray beam should be well modelled by 2d Gaussian
		\item \nb{Subtract black from grey images to give only spot response}
		\item fit 2d Gaussian with constraint on location
		\item If centre is on constraint boundary, compare constrained model to free model
		\begin{itemize}
			\item Improvement in RMSE suggests spot is off-centre
			\item No improvement suggests response is not elliptical
		\end{itemize}
	\end{itemize}
		\todo{Central spot: 14-10-09}
		\todo{Off-centre spot: 16-07-05}
		\todo{Non-elliptical response: Nikon panel MCT225}
\end{frame}
 
%-----------------------------------------------------------------------

% INCLUDE STANDARD DEVIATION AS ONE OF THE VARIABLES?

\subsubsection{Classification}

\begin{frame}{Classification}

	All variables are thresholded using the approach already outlined
	
	Pixelwise mean values are also divided into degrees of brightness \nb{or am I going to do this to all attributes?}	
	
	\todo{histogram of pixelwise mean values with thresholds marked}
	\todo{table of classifications. Give counts of pixels observed?}

\end{frame}
% \todo{table of classifications. Give counts of pixels observed?}

\subsection{Two case studies}

\subsubsection{Images from 14-10-09}

\begin{frame}{Images from 14-10-09}{Pixel values}
	\todo{black pw.m + hist + px identified}
	\todo{grey pw.m + hist + px identified}
\end{frame}

%-----------------------------------------------------------------------

\begin{frame}{Images from 14-10-09}{Median-smoothed residuals}
	\todo{black residuals + hist + px identified}
	\todo{grey residuals + hist + px identified}
\end{frame}

%-----------------------------------------------------------------------

\begin{frame}{Images from 14-10-09}{Model residuals}
	\todo{linear residuals + hist + px identified}
	\todo{spot residuals + hist + px identified}
\end{frame}
	
%-----------------------------------------------------------------------

\begin{frame}{Images from 14-10-09}{Pixels identified}
	\todo{plot of pixels identified (colour-coded \& with counts)}
	\nb{shading-corrected image?}
\end{frame}
	
\subsubsection{Images from 16-07-19}

\begin{frame}{Images from 16-07-19}{Pixel values}
	\todo{black pw.m + hist + px identified}
	\todo{grey pw.m + hist + px identified}
\end{frame}

%-----------------------------------------------------------------------

\begin{frame}{Images from 16-07-19}{Median-smoothed residuals}
	\todo{black residuals + hist + px identified}
	\todo{grey residuals + hist + px identified}
\end{frame}

%-----------------------------------------------------------------------

\begin{frame}{Images from 16-07-19}{Model residuals}
	\todo{linear residuals + hist + px identified}
	\todo{spot residuals + hist + px identified}
\end{frame}
	
%-----------------------------------------------------------------------

\begin{frame}{Images from 16-07-19}{Pixels identified}
	\todo{plot of pixels identified (colour-coded \& with counts)}
\end{frame}

\section{Multi-pixel features}

\subsection{Spots on beryllium window}

\begin{frame}[label = screen-spots]{Spots on beryllium window}
	\begin{itemize}
		\item Visible in some images as irregular dim spots
		\item Caused by flecks of superheated tungsten sticking to beryllium screen
		\item Can only be removed by replacing the beryllium window
		\item Apparent position likely to move between acquisitions
		\item Not always removed by shading correction
	\end{itemize}
		\todo{white image with spots: 14-10-09}
		\todo{shift in apparent position between acquisitions}
		\todo{shading-corrected image with spots: 14-10-09}
		\todo{link to appendix showing process of identification \hyperlink{ss-supplemental}{here}}
\end{frame}

%-----------------------------------------------------------------------

\begin{frame}{Screen spots in linear residuals}

	\begin{itemize}
		\item Not always visible - behaving like any object in field of view
		\item May appear as dim patch - thin enough to be invisible in white image, but not in grey?
		\item May appear with associated bright patch
		\item Will definitely not be removed by shading correction
		\item May indicate a pit in the tungsten target?
		\item Or just different diffraction at different power levels?
	\end{itemize}	 
		\todo{spots visible in linear residuals}
\end{frame}


%======================================================================

\subsection{Dense regions}

\begin{frame}{Dense regions of defects}
	\begin{itemize}
		\item Dense regions suggest a larger issue
		\item Tend to arise around edges of panels
		\item Can be identified by density filter
	\end{itemize}
	
	\todo{examples of dense regions: 160430 corners, loan panel}
	
\end{frame}

%-----------------------------------------------------------------------

\subsection{Linear defects}

\begin{frame}{Column defects}

\end{frame}

%-----------------------------------------------------------------------

\begin{frame}{Row defects}
	\begin{itemize}
		\item Two row defects observed in current loan panel
		\item Row 77: decreasing gradient R-L
			\todo{three transects as before}
		\hrule
		\item Row 1025: constant offset
			\todo{three transects as before}
		\item Assume that these must occur after data is transferred along columns, in readout register?
	\end{itemize}
\end{frame}

%-----------------------------------------------------------------------

\subsection{Cluster defects}

\begin{frame}{Cluster defects}

	
\end{frame}


%%%%%%%%%%%%%%%%%%%%%%%%%%%%%%%%%%%%%%%%%%%%%%%%%%%%%%%%%%%%%%%%%%%%%%%%%%%%%%%%%%%%%%%%%%%%%%%%%%%%%%%%%%%%%%%%%%%%%%%%%%%%%%%%%%%%%%%%%%%%%%%%%%%%%%%%%%%
\appendix
%%%%%%%%%%%%%%%%%%%%%%%%%%%%%%%%%%%%%%%%%%%%%%%%%%%%%%%%%%%%%%%%%%%%%%%%%%%%%%%%%%%%%%%%%%%%%%%%%%%%%%%%%%%%%%%%%%%%%%%%%%%%%%%%%%%%%%%%%%%%%%%%%%%%%%%%%%%

\section{Screen spots}

\begin{frame}[label = ss-supplemental]{Identification of screen spots}
	Supplemental content.
	Back to \hyperlink{screen-spots}{screen spots}.
\end{frame}

%%%%%%%%%%%%%%%%%%%%%%%%%%%%%%%%%%%%%%%%%%%%%%%%%%%%%%%%%%%%%%%%%%%%%%%%%%%%%%%%%%%%%%%%%%%%%%%%%%%%%%%%%%%%%%%%%%%%%%%%%%%%%%%%%%%%%%%%%%%%%%%%%%%%%%%%%%%
\end{document}
%%%%%%%%%%%%%%%%%%%%%%%%%%%%%%%%%%%%%%%%%%%%%%%%%%%%%%%%%%%%%%%%%%%%%%%%%%%%%%%%%%%%%%%%%%%%%%%%%%%%%%%%%%%%%%%%%%%%%%%%%%%%%%%%%%%%%%%%%%%%%%%%%%%%%%%%%%%