\documentclass{beamer}

%\usetheme{Singapore}
%\usetheme{Madrid}
\usetheme{default}

\hypersetup{pdfpagemode=FullScreen}		% pdf will automatically open in presentation mode

\title{The Life \& Death of Pixels}
%\subtitle{Optional Subtitle}

\author{Clair Barnes}

\institute[University of Warwick] % (optional, but mostly needed)
{
  Department of Statistics \& WMG\\
  University of Warwick
}

\date{Steering Group Meeting, September 20th 2016}


%======================================================================


\begin{document}

\begin{frame}
  \titlepage
\end{frame}

\begin{frame}{Outline}
  \tableofcontents
  % You might wish to add the option [pausesections]
\end{frame}

%-----------------------------------------------------------------------

\section{Identification of defects}

\subsection{Classification of abnormal pixels}

\begin{frame}{Abnormal pixel values}%{Optional Subtitle}
  \begin{itemize}
  \item Identification of abnormal pixels: histogram of black/grey/white values, with cropping to show the length of the tails
  \item Applying a simple 6-sigma cutoff doesn't work particularly well because data may not be symmetrical
  \item Need robust approach because of expected relatively large numbers of pixels with very high or very low values
  \item Straightforward approach: divide data at modal density (not necessarily modal value), take NMAD of each half, use 6-NMAD. If data were normally distributed, this would be equivalent to a 6-sigma approach, but more flexible when data is not normal.
  \item cutoffs quite close to body of data, so pick up maximum number of bad pixels. NB purpose here is not to create a bad pixel map for correction, it is to pick up as many abnormally-valued pixels as possible so that we can evaluate their behaviour.
  \item To identify pixels to add to bad pixel map, easier to go straight to shading correction (depending on purpose)
  \end{itemize}
\end{frame}

\begin{frame}{Types of defect}%{Optional Subtitle}
  \begin{itemize}
  		\item Strong correlation between defect residuals in grey \& black images (0.99, vs 0.44 for all pixels - although this is based on only one image, need to look more closely into it
  		\item scatterplot of error residuals in grey \& black, with correlation score, and abline with gradient 1
  		\item suggests that majority of errors aren't due to a problem in response, so much as a base offset from expected value (sometimes called a hot pixel in the literature - different names used here to allow easier distinction between degrees, for later on
  		\item Are these offset errors generally removed by shading correction?
  \end{itemize}
\end{frame}

\begin{frame}{Nonlinearity}
	Add in notes on nonlinear response of pixels, IF it needs to be included - not sure if this will add anything much!
\end{frame}

%-----------------------------------------------------------------------

\subsection{Multi-pixel features}

\begin{frame}{Column defects}
	
	\begin{itemize}
		\item Most notable errors are dark columns - `blocked' pixel
		\item Also bright \& dim columns, sometimes in combination - `leaking' or `draining' charge
		\item Do column errors ever develop, or are they essentially static?
		\item (Are offset errors ever a problem in the shading-corrected image?)
	\end{itemize}
\end{frame}

%-----------------------------------------------------------------------

\begin{frame}{Cluster defects}

	\begin{itemize}
		\item Discuss main direction of `spread' within defects: horizontal or vertical, or both?
		\item Clusters in conjunction with lines - how many lines have cluster roots, how many clusters are on a line?
	\end{itemize}

\end{frame}

%-----------------------------------------------------------------------

\begin{frame}{High-density regions}

	\begin{itemize}
		\item Areas of detector in which an extremely high density of pixels occurs
		\item Tend to occur in panel corners or edges
		\item Use in conjunction with spot response to identify potentially damaged regions 
	\end{itemize}

\end{frame}

%-----------------------------------------------------------------------

\subsection{Non-pixel features}

\begin{frame}{Spots on beryllium window}

	\begin{itemize}
	
		\item Causes of screen spots
		
		\item Identification of screen spots
		
		\item Proportion of screen identified (esp. on day when this was noted by Jay), size of screen spots identified
		
		\item Why is this a problem? - not removed by shading correction, presumably expensive to fix, 
		
		\item Assess number of spots not cleared by shading correction: what \% of pixels are affected?
	
	\end{itemize}
	
\end{frame}

%-----------------------------------------------------------------------

\begin{frame}{Full-row and full-column defects}

	\begin{itemize}
		\item Full-column defects
		\begin{itemize}
			\item Observed mainly in `loan' panel
			\item BGW in different order to those columns where line end is observed
			\item Suspect that this is due to a problem in readout.
		\end{itemize}
		\item Full-row defects
		\begin{itemize}
			\item Only observed in `loan' panel
			\item Two rows: one has constant offset, one has steady decline in gradient from right to left
			\item Since panels are read out separately, can only assume that this is a problem in the readout array. Thoughts?
		\end{itemize}
	\end{itemize}
	
\end{frame}

%-----------------------------------------------------------------------

\subsection{General detector health: response to spot}

%-----------------------------------------------------------------------

\begin{frame}{Non-circular response to x-ray source}{Assessing overall panel response}
  \begin{itemize}
  	\item X-rays are emitted in a cone, so expect to see a circular response, well modelled by a 2d Gaussian
  	\item Deviations from that pattern may indicate less sensitive regions within detector, or an off-centred spot
  	
  	\item Fit an elliptical 2d Gaussian (x \& y independent) with focus constrained to be in centre
  	\item If centre is on constraint boundary, compare constrained model to a free model
  	\begin{itemize}
  		\item Improvement in RMSE suggests spot is off-centre
  		\item No improvement suggests no elliptical pattern to be found - indicating that there are unexpected regions of high or low values in the detector
	\end{itemize}
  \end{itemize}
\end{frame}

\begin{frame}{Non-circular response to x-ray source}
  \begin{itemize}
  	\item Why is this important?
	\begin{itemize}
		\item Power settings are selected based on nominal value, which assumes normal response (see bimodal plot of MCT225, or v skewed 160705) - if response is not 2d Gaussian, may need to revise power settings to get appropriate value in given region of detector
		\item If heavily desensitized, detector may eventually need to be refurbished - tracking degree of deviation from expected response 
	\end{itemize}		 
  \end{itemize}
\end{frame}


%======================================================================


\section{Persistence of defects}

\subsection{Short-term persistence of defects}
\begin{frame}{Persistence of defects}

	\begin{itemize}
		\item Proportion of defects (per type) remaining defective in following acquisition
		\item Of defects that persist from one acquisition to next, what proportion persist for a third, etc...
		\item Proportion of recurrent/intermittent defects (limited to single detector)
		\item Did any pixels remain unhealthy after the two refurbishments? (What happens at refurbishment, anyway?)
	\end{itemize}
\end{frame}

\subsection{Progression of defects}
\begin{frame}{Progression of defects}
	\begin{itemize}
		\item Transition matrix (non-instantaneous) between observed states: mean and SD
		\item Once pixels reach a given state, do they tend to stay there, or do pixels ever return to a healthy state? Is there a `point of no return'?
	\end{itemize}

\end{frame}

\begin{frame}{Clusters of defective pixels}
	\begin{itemize}
		\item Do clusters ever shrink?
		\item Total cluster value, as well as size
		
	\end{itemize}

\end{frame}


%======================================================================


\section{Spatial distribution of defects}
\subsection{Patterns in spatial distribution of defective pixels}

\begin{frame}{Spatial distribution of defects}

	\begin{itemize}
		\item Are defects more likely to occur in some regions of the panel than others?
		\item Is this related to size/severity of defect at all?
		\item Is inhomogeneous distribution of defects related in any way to surface of dark image?
	\end{itemize}

\end{frame}

%======================================================================


\section*{Summary}

\begin{frame}{Summary}

  \begin{itemize}
  \item
  	Benefits
  	\begin{itemize}
    \item
    		Detector `health check' - independent assessment of possible problems
    	\item
    		Acceptable tolerances can be tailored to operator's requirements
  	\item
    		Assess rate of deterioration - better planning for refurbishment/replacement
    	\end{itemize}
  \end{itemize}
  
  \begin{itemize}
  \item
    Future work
    \begin{itemize}
    \item
      More extensive study with more panels, each examined over a period of time 
    \item
    	  Currently lacking a `gold standard' - obtain sample of new panels to test assumptions of `healthy' behaviour \& set more informed thresholds
    \end{itemize}
  \end{itemize}
  
\end{frame}



\end{document}
