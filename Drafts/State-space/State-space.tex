\documentclass[10pt,fleqn]{article}
\usepackage{/home/clair/Documents/mystyle}

%----------------------------------------------------------------------
% reformat section headers to be smaller \& left-aligned
\titleformat{\section}
	{\normalfont\bfseries}
	{\thesection}{1em}{}
	
\titleformat{\subsection}
	{\normalfont\bfseries}
	{\thesubsection}{1em}{}

%======================================================================
\newcommand{\q}[1]{
	\begin{minipage}{\textwidth}
	\textcolor{black}{
					\begin{tabular}{p{0.01\textwidth}p{0.95\textwidth}}
						$\square$ & #1
					\end{tabular}
					}
	\end{minipage}
}

%======================================================================

\begin{document}

\section*{QUESTIONS / CONSIDERATIONS}




\q{When compiling Markov model, treat non-responsive/dead columns (apart from head? Need to define column head) as censored data. We don't know what state they are in, we can simply no longer access those pixels.}

\q{What shall we do with pixels on a bright line? Treat as a separate group (on the assumption that `bright line' behaviour will subsume individual pixel behaviour) or adjust by observed column offset \& reclassify? Suspect that most representative approach is to treat line head (end furthest from midpoint/panel edge) as a distinct category, then treat pixels along the affected column as any other pixel. Increased current is most likely occurring in readout, not in pixels themselves. This also ties in with treatment of dead lines as non-responsive.}

\newpage

\section{Thresholds}

\subsection{Plot each threshold (in terms of $\sigma$ vs \% of problematic pixels identified}

\subsection{Relate each type of bad pixel to behaviour in shading correction}

\subsection{Tabulate best-case vs worst-case scenario for classification (essentially, switching sort order before removing duplicates)}

\section{State spaces observed in a single acquisition}

\href{http://tex.stackexchange.com/questions/178904/use-datatool-to-read-a-row-from-a-csv-file-then-use-the-variables-in-the-docume}{Read data into variables using \texttt{datatool}} \\
\href{http://www.texample.net/tikz/examples/state-machine/}{State machine in \texttt{tikz}}

\subsection{Per-pixel state space diagram}

\subsection{Stability of rate changes: constant or increasing? (Or decreasing, even?)}

\todo{Plot number of pixels identified as each type of bad over all 12 acquisitions: is trend linear?}

\subsection{State space diagram showing features}


%\newpage
%\printbibliography
\end{document}