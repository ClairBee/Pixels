\documentclass[10pt,fleqn]{article}
\usepackage{/home/clair/Documents/mystyle}

%----------------------------------------------------------------------
% reformat section headers to be smaller \& left-aligned
\titleformat{\section}
	{\normalfont\bfseries}
	{\thesection}{1em}{}
	
\titleformat{\subsection}
	{\normalfont\bfseries}
	{\llap{\parbox{1cm}{\thesubsection}}}{0em}{}
	
%----------------------------------------------------------------------
% SPECIFY BIBLIOGRAPHY FILE & FIELDS TO EXCLUDE
%\addbibresource{bibfile.bib}
%\AtEveryBibitem{\clearfield{url}}
%\AtEveryBibitem{\clearfield{doi}}
%\AtEveryBibitem{\clearfield{isbn}}
%\AtEveryBibitem{\clearfield{issn}}

%----------------------------------------------------------------------
% define code format
\lstnewenvironment{code}[1][R]{			% default language is R
	\lstset{language = #1,
        		columns=fixed,
    			tabsize = 4,
    			breakatwhitespace = true,
			showspaces = false,
    			xleftmargin = .75cm,
    			lineskip = {0pt},
			showstringspaces = false,
			extendedchars = true
    }}
    {}
    
%======================================================================

\usepackage{fancyhdr}

\fancyhf{}
\fancyhead[R]{\textbf{23-Feb-2016}}
\pagestyle{fancy}

\addtolength{\topmargin}{-0.5cm}
\addtolength{\textheight}{1.4cm}

\renewcommand{\headrulewidth}{0pt}

\begin{document}

\subsection*{Since last week}

\begin{itemize}

\item
Brief literature review carried out. Some suggestions for thresholding are listed below.

\end{itemize}

\subsection*{Current questions/considerations}

Thresholding ideas/possibilities:

\begin{itemize}
\item
For each pixel, apply signed rank test to difference between pixel and its neighbours (or simply determine if centroid is max/min of block) - a pixel that is max/min of all its neighbours is probably either hot or dead.

\item
Set threshold for bad pixels based on difference between $Q_{99.9}$ (say) and modal value

\item
\end{itemize}

\subsection*{Next steps}

\begin{itemize}

\item
Create a set of functions to identify `bad' pixels. Begin by using thresholds/definitions given in manual, but make functions flexible in order to update parameters later.

\item
In particular, plot daily time series of pixels with high daily SD - these may be showing short-term 'flickering'/'blipping' behaviour. (Those investigated so far seem fairly constant in the short term)

\item
For all `bad' pixels identified, plot daily pixelwise mean for full sequence of measurements

\item
For each panel, look at pixelwise mean \& SD - should be a relationship between the two, which will differ across the panels (intra-panel dependency)


\end{itemize}

%\newpage
%\printbibliography
\end{document}