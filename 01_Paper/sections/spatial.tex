\providecommand{\main}{..}					% fix bibliography path
\documentclass[../../IO-Pixels.tex]{subfiles}
    
   
%======================================================================

\begin{document}

\section{Spatial analysis}
%----------------------------------------------------
\begin{outline}
Spatial statistics 101 (point patterns, F-, G-, K-functions etc)

Parametric model basics

Parametric model to adjust for spatial inhomogeneity

CSR testing using K-functions etc

Patterns: singletons, pairs, clusters, lines, bright lines etc

\\
Above 

- for all malfunctioning pixels and 

- for subsets (dead, hot, dim etc)

- also for spots on screen
- how far do spots on screen move (on average) in successive images?
- can movement of spots on screen be related to position of spot centre at all?

\\
Modeling: e.g. Neyman-Scott type, Matern, Thomas... models

Intermediate discussion

\\
+ Compare spatial distribution across different threshold types: `official' threshold/ nonlinear px/ our thresholds/ shading-corrected median differences / standard deviations. Hope to see similar intensity across all bad pixel maps
\end{outline}
%----------------------------------------------------

\addfigure{Show G-, F- and normed K-functions for bad pixel map incl. all pixels, and feature roots only (cluster \& line bodies removed)}
\addfigure{Model spatial inhomogeneity as quadratic trend \& as kernel density. Show K-functions of both}

\addfigure{check quadrat plots across subpanels, but also double, half- \& quarter-subpanels (may need to use MC if expected n is too small) - is significant difference only across subpanels?}

\addfigure{Consider moving thresholds between dim/dark, bright/hot etc; does this change anything?}
\end{document}
