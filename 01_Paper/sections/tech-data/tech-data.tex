\providecommand{\main}{../..}					% fix bibliography path
\documentclass[\main/IO-Pixels.tex]{subfiles}
    
   
%======================================================================

\begin{document}

\section{Technical backgound \& data}
%----------------------------------------------------
\begin{outline}
Detector and X-ray technology

Details on data collections: 

- pilot data: bad pixel maps, mean and SD files)

  Perkin Elmer definitions of underperforming pixels and discussion
  
- experimental data: protocol for white image test data, usage data
\end{outline}
%----------------------------------------------------

All of the detector images used in the study were cropped to some degree. However, all images are taken from a panel of the same dimensions (2048 pixels square, consisting of two rows of 16 subpanels, each of 128 $\times$ 1024 pixels). For ease of visual comparison, all images have been padded to a $2048 \times 2048$ square so that they all occupy their true position, relative to the detector edges, within the plot area.
  
\addfigure{Describe shading correction. For our purposes, to explore the effect of defective pixels on the shading-corrected, we will treat the grey image as the `true' image \nb{Check with Jay: is this reasonable to do?}}

\begin{table}[!ht]
\caption{\nb{Table works fine in TeX, but can't read CSV here - need to find out why} Summary of images in data set, with date/time acquired, panel model, and power settings applied when obtaining each image (kV = voltage, uA = current; exp. = exposure time in $\mu s$ \nb{?}) The total power input is also calculated (in arbitrary units \nb{?}) for ease of comparison. \\ \footnotesize{The first 6 image sets comprise the `pilot data' set, the 14 sets acquired between October 2014 and July 2016 the `main data' set. All of these images were acquired from the same detector panel, which was refurbished on the dates indicated, and was replaced with the `loan' panel after this date due to a malfunctioning readout sensor affecting a whole subpanel. The remaining images were acquired singly from detectors kindly loaned from other institutions. All images were acquired with a Reflection 225 source and tungsten source target. \nb{different gain settings were used in obaining some of the images - check with Jay if this should be included - also, should we mention the different operators?}}}

\resizebox{\textwidth}{!}{%
	\csvreader[tabular = cc|c|c|c cc|c|ccc|c|l, head to column names, late after line =, 
				before line = \ifthenelse{\equal{\hlflag}{}}{\\}{\\\hline}, before first line =,
				table head = Image & Panel & &	 & \multicolumn{4}{c|}{Grey} & \multicolumn{4}{c|}{White} & \\
				  ref  & ref & Timestamp & Model & \multicolumn{4}{c|}{\footnotesize{kV $\cdot$ uA $\cdot$ exp.} $=$ power} & \multicolumn{4}{c|}
				  {\footnotesize{kV $\cdot$ uA $\cdot$ exp.} $=$ power}  & Comments\\\hline]%
		{\main/fig/tech-data/Detector-details.csv}{}%
		{\code & \panelid & \footnotesize{\timestamp} & \model & \footnotesize{\gkV} & \footnotesize{\guA} & \footnotesize{\gexp} & \gpower & 
			\footnotesize{\wkV} & \footnotesize{\wuA} & \footnotesize{\wexp} & \wpower & \footnotesize{\notes}}%
	}
\end{table}


\end{document}
