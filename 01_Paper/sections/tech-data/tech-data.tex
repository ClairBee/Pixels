\providecommand{\main}{../..}					% fix bibliography path
\documentclass[../../IO-Pixels.tex]{subfiles}
    
\graphicspath{{\main/fig/tech-data/}}			% fix graphics path
%======================================================================

\begin{document}

\section{Technical backgound \& data}
%----------------------------------------------------
\begin{outline}
Detector and X-ray technology

Details on data collections: 

- pilot data: bad pixel maps, mean and SD files)

  Perkin Elmer definitions of underperforming pixels and discussion
  
- experimental data: protocol for white image test data, usage data
\end{outline}
%----------------------------------------------------

All of the detector images used in the study were cropped to some degree. However, all images are taken from a panel of the same dimensions (2048 pixels square, consisting of two rows of 16 subpanels, each of 128 $\times$ 1024 pixels). For ease of visual comparison, all images have been padded to a $2048 \times 2048$ square so that they all occupy their true position, relative to the detector edges, within the plot area.
  
\addfigure{Describe shading correction. For our purposes, to explore the effect of defective pixels on the shading-corrected, we will treat the grey image as the `true' image \nb{Check with Jay: is this reasonable to do?}}

\newcommand*{\OptionalHline}{\edef\myflag{\hlflag}\ifdefvoid{\myflag}{}{\vspace{-10pt}\\\hline}}

\begin{footnotesize}
	\csvreader[tabular=cc|c|c|c cc|c|ccc|c|l, head to column names=true, before line = \OptionalHline,
	table head = Image & 	   &			  &		  & \multicolumn{4}{c|}{Grey} & \multicolumn{4}{c|}{White} & \\
				  ref  & Panel & Timestamp & Model & \multicolumn{4}{c|}{kV $\cdot$ uA $\cdot$ exp. $=$ power} & \multicolumn{4}{c|}{kV $\cdot$ uA $\cdot$ exp. $=$ power}  & Comments\\\hline]%
		{\main/fig/tech-data/Detector-details.csv}{}%
		{\code & \panelid & \timestamp & \model & \gkV & \guA & \gexp & \gpower & \wkV & \wuA & \wexp & \wpower & \notes}%
	\end{footnotesize}


\end{document}
