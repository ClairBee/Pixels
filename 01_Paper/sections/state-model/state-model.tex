\documentclass[../../IO-Pixels.tex]{subfiles}
    
   
%======================================================================

\begin{document}


\section{Observable states of pixels}


%===========================================================================================================


\subsection{Description of defective pixel types}

\todo{Glossary of bad pixel definitions \& thresholds applied}

\todo{Description of how screen spots are identified \& their effects removed}



%===========================================================================================================


\subsection{Behaviour of each defective pixel type under shading correction}

\todo{Which types of bad pixels are of greatest concern? ie. which need to be retained in a bad pixel map, which are only tracked to assess performance/deterioration of detector?}


%===========================================================================================================


\subsection{Observed state spaces \& transitions}

\todo{Note computational difficulty of fitting a likelihood model to 4 million pixels to estimate `true' instantaneous transition probabilities}

\todo{Model is necessarily only descriptive. However, we have 12 observed acquisitions, with 11 sets of transitions, which allow us to observe some movement between states and to propose a plausible set of state transitions}

\todo{Flow chart of state changes, with transition probabilities \& their standard deviations}

\todo{Note that not all transitions are shown in the flow chart, even though they may have appeared in a transition matrix for a pair of acquisitions. We assume that the pixel has to pass through the intermediate states between observations}

\todo{Plot observed transitions over time, demonstrating that rates of movement between states do not increase or decrease}

%===========================================================================================================


\subsection{Larger features}

\todo{Clusters of bright/hot pixels}

\todo{Clusters involving non-responsive pixels?}

\todo{Columns of bright pixels}

\todo{Columns of non-responsive (inaccessible) pixels}

\todo{Remark on hypothesized development of bright pixels into bleeds into blocked channels. However, the project was too short to observe this progression; hopefully further work will investigate this further}

%===========================================================================================================


\subsection{Rate of detector deterioration: real time vs usage time}

\todo{Radiation damage is known to cause bright defects at an increased rate (see Hubble ISR re hot pixel) so we might expect rate of increase of certain types of pixels to be correlated with usage}

\todo{Compare increase in defects over chronological time vs over usage time \nb{How to calculate/enumerate usage? Is it as simple as kV $\times$ uA $\times$ power $\times$ exposure time $\times$ no. frames?}}

\todo{Also: any link between periods of particularly heavy usage and appearance of screen spots?}

%\newpage
%\printbibliography

\end{document}