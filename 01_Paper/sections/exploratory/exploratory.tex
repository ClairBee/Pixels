\providecommand{\main}{../..}					% fix bibliography path
\documentclass[\main/IO-Pixels.tex]{subfiles}
   
\graphicspath{{\main/fig/exploratory/}}			% fix graphics path
%======================================================================

\begin{document}

\section{Exploratory analysis of experimental data}
%----------------------------------------------------
\begin{outline}
Means: overall and spatial distribution

SDs: overall and spatial distribution

Subpanels

Refined definition of states in the experimental data; thresholds

\\
+ Screen shots

+ Edge cropping

+ Parametric description of panel (esp. black images)

+ Comparison of images from several detectors
\end{outline}
%----------------------------------------------------
Unless stated otherwise, image plots are shaded to highlight changes in value close to the mean; extreme values are less finely binned.

\subsection{Description of data}
\addfigure{Images of black, grey \& white pixelwise means}
\addfigure{Images of black/grey/white pixelwise SD? Or scatterplot vs pixelwise mean?}
\addfigure{Images of quadratic trend model for black, grey \& white (grey \& white with dark image removed)}
\addfigure{Images of linear gradients across subpanels}
\addfigure{Boxplots of values in each subpanel?}

\subsection{Other considerations: screen spots and edge effects}
%----------------------------------------------------
% screen spots in shading correction & shifting in successive images
\begin{figure}[!ht]
\caption{Screen spots visible in the images acquired on 14-10-09 and successive dates (spots are circled for ease of identification)}
\centering

\begin{subfigure}[t]{0.32\textwidth}
\caption{White pixelwise mean image}
\includegraphics[scale=0.3]{pwm-image-141009-white}
\end{subfigure}
%
\begin{subfigure}[t]{0.32\textwidth}
\caption{White shading-corrected image}
\includegraphics[scale=0.3]{sc-image-141009-white}
\end{subfigure}
%
\begin{subfigure}[t]{0.32\textwidth}
\caption{Shift in apparent position}
\includegraphics[scale=0.3]{spots-overplotted}
\end{subfigure}
\end{figure}
%----------------------------------------------------
\addfigure{Need to decide how to handle screen spots. Do they need to be removed from shading-corrected image (as suggested by plot above) or is the shading correction still effective enough that we can afford to ignore them (as suggested by the actual approach taken)? In SC image for 141009, 99.9\% of values lie between 19624 and 20225 (range of 600px)}

Screen spots that have appeared since the previous acquisition can be easily identified by subtracting the new grey or white image from the previous acquisition. However, the `shadow' cast by the spots on the detector will shift slightly in different acquisitions due to small differences in the position or angle of the x-ray source, so spots need to be identified and corrected for at each acquisition.

\subsection{Thresholding \& definitions of bad pixels}

\addfigure{Histograms showing thresholds of extreme-valued pixels}
\addfigure{Linear regression of white vs black \& grey images, showing thresholding by residuals}
\end{document}
